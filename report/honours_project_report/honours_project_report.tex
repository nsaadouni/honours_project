
% to choose your degree
% please un-comment just one of the following
\documentclass[bsc,frontabs,twoside,singlespacing,parskip,deptreport]{infthesis}     % for BSc, BEng etc.
% \documentclass[minf,frontabs,twoside,singlespacing,parskip,deptreport]{infthesis}  % for MInf

\usepackage{pdfpages}
\usepackage{float}% If comment this, figure moves to Page 2
\usepackage{subcaption}
\usepackage{tabulary}
\usepackage{listings}
%\usepackage{hyperref}

% This is for the Verbatim package
\usepackage{fancyvrb}
\usepackage{bera}

\usepackage[toc,page]{appendix}

\begin{document}

\title{Finding Vulnerabilities in Low-Level Protocols}

\author{Nordine Saadouni}

% to choose your course
% please un-comment just one of the following
%\course{Artificial Intelligence and Computer Science}
%\course{Artificial Intelligence and Software Engineering}
%\course{Artificial Intelligence and Mathematics}
%\course{Artificial Intelligence and Psychology }   
%\course{Artificial Intelligence with Psychology }   
%\course{Linguistics and Artificial Intelligence}    
\course{Computer Science}
%\course{Software Engineering}
%\course{Computer Science and Electronics}    
%\course{Electronics and Software Engineering}    
%\course{Computer Science and Management Science}    
%\course{Computer Science and Mathematics}
%\course{Computer Science and Physics}  
%\course{Computer Science and Statistics}    

% to choose your report type
% please un-comment just one of the following
%\project{Undergraduate Dissertation} % CS&E, E&SE, AI&L
%\project{Undergraduate Thesis} % AI%Psy
\project{4th Year Project Report}

\date{\today}


\abstract{
Basic example of an abstract (this will be changed)\\

Smart cards are used commercially and within industry for authentication, encryption, decryption, signing and verifying data. This paper aims to look into how the smart card interacts with an application at the lower level. PKCS\#11 (public key cryptography system?) is the standard that is implemented at the higher level and then broken down into command/response pairs sent as APDU traffic to and from the smart card. It is the APDU low-level protocol that will be analysed to see if any vulnerabilities are present with regard to the smart cards tested.\\
}

\maketitle


\section*{Acknowledgements}
Acknowledgements go here.
% Myrto, Andriana, Parents
% MSC student? 

\tableofcontents

%\pagenumbering{arabic}

%---------------------------------------------------------------------------------------------------------------------------------------------------

\chapter{Introduction}
Smart-cards are formally known as integrated circuit cards (ICC), and are universally thought to be secure, tamper-resistant devices. They store and process, cryptographic keys, authentication and user sensitive data. They are utilised to preform operations where confidentiality, data integrity and authentication are key to the security of a system.\\

\noindent Smart-cards offer what seems to be more secure methods for using cryptographic operations. (And should still provide the same level of security that would be offered to un-compromised systems, compared to those that are compromised by an attacker). This is partly due to the fact that the majority of modern smart-cards have their own on-board micro-controller, to allow all of these operations to take place on the card itself, with keys that are unknown to the outside world and stored securely on the device. Meaning the only actor that should be able to preform such operations would need to be in possession of the smart-card and the PIN/password. In many industries, for applications such as, banking/ payment systems, telecommunications, healthcare and public sector transport, 
%\begin{itemize}
%\item Banking/ payment
%\item E-commerce
%\item Sim cards/ telecommunications
%\item Healthcare
%\item Public transport
%\end{itemize}
smart-cards are used due to the security they are believed to provide.\\


\noindent The most common API (application programming interface) that is used to communicate with smart-cards is the RSA defined PKCS\#11 (Public Key Cryptography Standard). Also known as 'Cryptoki' (cryptographic token interface, pronounced as 'crypto-key'). The standard defines a platform-independent API to smart-cards and hardware security modules (HSM). PKCS\#11 originated from RSA security, but has since been placed into the hands of OASIS PKCS\#11 Technical Committee to continue its work (since 2013). [reference wikipedia PKCS\#11].\\


% Note: decide on a figure to use here, 10 OR 15 years!
% This paragraph needs to be re-written!
\noindent In the previous 10-15 years, literature has shown a great deal of research into the examination of the PKCS\#11 API, and the security it provides. Yet little attention has been paid to that of the lower-level communication (command-response pairs), in which the higher level API is broken down into. It is this area that we wish to research within this paper. The reasoning is simple. If we cannot trust the security of the low-level commands that implement the high level API functions, then in turn we cannot trust the security of the high level API. This is analogous to C code being complied down to binary data to be operated on by the CPU. The addition of two integers cannot be considered correct in C, unless the corresponding binary instructions sent to the CPU are correct as well.  

% maybe not 
The research of the low-level communication will take two forms. 
\begin{enumerate}
\item An analysis of the raw communication between PC and Smart-card for all API function calls
\item 
\end{enumerate}


Before we move into the above analysis, supporting material must be introduced. This the rest of this paper will be organized as follows.



%---------------------------------------------------------------------------------------------------------------------------------------------------

\chapter{Background}

In this chapter we give background knowledge on two essential standards that are required for the use of the contact smart-card we analyse in this paper. These two standards are PKCS\#11 and ISO 7816.

\section{PKCS\#11}
This is the API that each card manufacture implements themselves.

\subsection{Key Object}
\subsection{Attributes}
\hskip-2.0cm\begin{verbatim}
template = (\\
	(CKA\_CLASS, LowLevel.CKO\_SECRET\_KEY), [private, public, data, cert (X.509)]
	(CKA\_KEY_TYPE, LowLevel.CKK_DES), [AES,DES,RSA,ECC]
	(CKA\_LABEL, label), [name]
	(CKA\_ID, chr(id)), [id]
	(CKA\_PRIVATE, True), [requires authentication]
	(CKA\_SENSITIVE, True), [cannot be extracted unencrypted]
	(CKA\_ENCRYPT, True), [can be used for encrypting data]
	(CKA\_DECRYPT, True), [can be used for decrypting data]
	(CKA\_SIGN, True), [[can be used for signing data]
	(CKA\_VERIFY, True), [can be used for verifiying data]
	(CKA\_TOKEN, True), [can be stored permanently on the device]
	(CKA\_WRAP, True), [can encrypt a key to be extracted]
	(CKA\_UNWRAP, True), [can decrypt an encrypted key]
	(CKA\_EXTRACTABLE, False)) [is allowed to be extracted from the device]
\end{verbatim}
\subsection{Most Common Functions}


\section{ISO 7816}
This is the international standards organization that defines the low level communication protocol between smartcards and the API on the computer.
\subsection{Command Structure}
\subsection{Response Structure}
\subsection{Inter-Industry/ Proprietary}
\subsection{Most Common Commands}

\begin{table}[H]
\hskip-2.5cm\begin{tabular}{|c|c|c|c|c|c|c|l|}
\hline
CLA & INS & P1 & P2 & Lc  & Data Field & Le & Description\\
\hline
00  & -   & -  & -  & -   & -          & -  & Inter-industry (II)\\
80  & -   & -  & -  & -   & -          & -  & Proprietary (P)\\
Xc  & -   & -  & -  & -   & -          & -  & Secure Messaging (SM)\\
\hline
00  & 84  & 00 & 00 & -   & -          & L  &  (II) Get Challenge [\# bytes = L]\\
00  & b0  & X  & X  & -   & -          & L  &  (II) Read Binary [\# bytes = L] \\
00  & c0  & 00 & 00 & -   & -          & L  &  (II) Get Remaining Bytes\\
00  & d6  & X  & X  & L   & X          & -  &  (II) Update Binary\\
00  & e0  & X  & X  & L   & X          & -  &  (II) Create Binary\\
00  & 47  & 00 & 00 & L   & params     & -  &  (II) Generate RSA KeyPair\\
00  & e4  & 00 & 00 & 00  & -          & -  &  (II) Delete File\\
00  & 2a  & 82 & 0e & L   & X          & 00 &  (II) Encrypt Data \\
00  & 2a  & 80 & 0e & L   & X          & 00 &  (II) Decrypt Data \\
00  & 2a  & 9e & 12 & L   & X          & 00 &  (II) Sign Data \\
00  & 2a  & 80 & 0a & L   & X          & 00  & (II) Unwrap Key (RSA\_PKCS)\\
\hline
80  & a4  & 08 & 00 & L   & FL         & -  & (P) Select File [FL = file location; L = len(fl)]\\
80  & a4  & 00 & 00 & -   & FL         & -  & (P) Select File [append previous path]\\
80  & a4  & 08 & 0c & L   & FL         & -  & (P) Read File Control Parameters\\
80  & 20  & 00 & 00 & 10  & R          & -  & (P) Verify PIN [R = Response]\\
80  & 28  & 00 & 00 & 04  & 00 00 00 20& -  & (P) Clear Security Status\\
80  & 30  & 01 & 00 & 00  & -          & -  & (P) List Files\\
80  & 48  & 00 & 80 & 00  & -          & -  & (P) Get Cards Public Key (genkey)\\
80  & 48  & 00 & 00 & 00  & -          & -  & (P) Get Cards Public Key (genkeypair)\\
80  & 86  & 00 & 00 & L   & X          & 00 & (P) Open Secure Messaging\\
80  & 86  & ff & ff & -   & -          & -  & (P) Close Secure Messaging\\
\hline
90  & 32  & 00 & 03 & ff  & X          & -  &  Reallocate Binary 256 bytes\\
80  & 32  & 00 & 03 & L   & X          & -  &  Reallocate Binary L bytes (changes file)\\
\hline
\end{tabular}
\caption{Common APDU Commands}
\end{table}
\subsection{File Structure}





%---------------------------------------------------------------------------------------------------------------------------------------------------

\chapter{Cryptographic Operations}

In this chapter we describe different cryptography standards, and give a brief overview of the terminology and maths used behind some of the types of cryptography. This will be used to aid our explainations in later chapters that regard the cards functionality and cryptographic operations used to attempt the secure transfer of sensitive information.

\section{Hash Function}

\noindent explain this in an overview term



\section{Asymmetric Encryption}

public and private keys for communicating over an insecure channel

\subsection{RSA}
Chinese Remainder Theorem RSA Keys

\section{Symmetric Encryption}

\noindent explain this in an overview term

\subsection{DES}
\subsection{Triple DES}
\subsection{AES}

% might go against those headings and just list them in the explanations
\subsection{ECB Mode}
\subsection{CBC Mode}


\section{Message Authentication Codes}

\noindent explain what a message authentication code is, what its used for\\
\noindent (wikipedia have good diagrams and explanations of this!!)

\subsection{Hash Based - Message Authentication Codes (HMAC)}
\subsection{Cryptographic Based - Message Authentication Codes (CMAC)}

\section{One Time Passwords}

\noindent What is a one time password $\rightarrow$ what forms are there?

\subsection{Hash Based - One Time Passwords (HOTP)}
\subsection{Time Based - One Time Passwords (TOTP)}




%---------------------------------------------------------------------------------------------------------------------------------------------------

\chapter{Tools}



\section{PCSC-lite}

\section{Virtual Smart-Card}
frank something
Explain how this creates a virtual smartcard reader
This is then used to run API functions, and the commands are relayed to the actual card reader and therefore card.


\section{Man in The Middle (MiTM)}





%---------------------------------------------------------------------------------------------------------------------------------------------------

\chapter{Related Work / Literature Review}

This will be a brief chapter and will discuss all of the research I have conducted.\\
Mainly regarding PKCS\#11 API attacks due to the small amount of literature that is available for APDU level attacks
I shall also explain why some of the attacks are not able to be conducted on the particular card I am reviewing






%---------------------------------------------------------------------------------------------------------------------------------------------------

\chapter{PKCS \#11 Functions - APDU analysis}

In this chapter we analyse the APDU traces of 9 functions from the API. C\_sign and C\_verify were not included in this analysis as from analysis of the traces from the previous work on this card it was clear that both of these functions operated in a similar manner to C\_encrypt and C\_decrypt. C\_createObject was also omitted form this analysis, as the function is not used for security operations, and rather for the creation and storing of certificates and data. The analysis will include a step by step guide of how the functions are broken down into command response pairs at the APDU level, and also suggest methods for attacking the card using this knowledge. 

All of the traces are provided in appendix B, and have been shortened to only include the parts which we deem to be the most important to the evaluation of how each function is broken down into APDU command response pairs. However, the full traces for each function are given within the project directory. In the analysis to come we will refer to appendix B, with the corresponding command response pair in brackets to allow ease of locating the steps we analyse.


\section{Initialization}
The first process before anything can occur on the card is the initialization. This occurs as soon as a session is opened with the card. The process has the following steps:
\begin{enumerate}
\item Open the laser PKI file and select the applet (B.1 - command 1)
\item Open the file and send to the API the cards serial number (B.1 - command 4,5)
\item Open a second file and send to the API another serial number (B.1 - command 8,9)
\end{enumerate}

None of these traces reveal sensitive information. But is required in order for the card to be able to operate and communicate with the API. The API is a generic library that works with many different hardware security modules created by the 'Athena smartcard solutions'. Hence this information is required by the library in order for it to operate correctly for the specific type of card.

\section{C\_login}
The login function has 5 main components. The API locates the file control parameters to check if the retry counter is greater than 0. In the APDU trace this is highlighted in bold and has a value of AA. From there the PIN file is selected, a challenge is asked for by the card and a response is calculated from the knowledge of the PIN and the challenge. Once this is completed the security status of the card is cleared. This is done because every security operation requires a new login by the API. While the user only logs in once, at the APDU layer this occurs many times. These steps are listed below:\\

\begin{enumerate}
\item Open file control parameters for the PIN file (B.2 - command 10)
\item Open the PIN file (B.2 command 11)
\item Request challenge from card (B.2 - command 12)
\item Verify PIN (B.2 - command 13)
\item Clear security Status (B.2 - command 20)\\
\end{enumerate}

From evaluating multiple traces of the login, it is clear that the response calculated seems to have an element of randomness. From studying the traces alone, the method of calculating the response cannot be intuitively determined.

While no immediate sensitive information is revealed in this trace as the challenge-response algorithm used for verifying a users PIN hides its value, successfully reverse engineering the protocol will allow an attacker to calculate the users PIN given successful login trace.

Furthermore, if an attacker can find a method for changing the file control parameter, it might be possible to reset the retry counter at the APDU level to allow an unlimited number of attempts of logging in.
\section{C\_findObject}
\textit{[For this trace to show meaningful information we pre-loaded a triple des key onto the card. The label of the key was 'des3' with id '01']}

The findObjects functions is split into 3 main components. First all the attribute files are listed that are stored on the card. The card has a file named 'cmapfile' that stores these locations. Using the data saved in this file, the API can locate the directory of each attribute file to load it into the API. A final file is opened that is zeroized, which dictates that no more attribute files for keys are present on the card. These steps are listed below:\\

\begin{enumerate}
\item Selects and opens 'cmapfile' [List files command]. This stores all file locations for attributes of the keys (B.3 - command 32, 33, 34)
\item Based on those locations, finds and opens the attribute file for the des3 key (B.3 - command 36)
\item And also opens the last attribute file to determine there are no more keys (B.3 - command 38, 39)\\
\end{enumerate}

Due to key and attribute files being stored separately, this function call reveals no sensitive information. Only the attributes are opened and listed at the APDU layer. These same attributes can be printed at the API layer as well.

A possibility for an attack is present here. The ISO 7816 'UPDATE BINARY' command can be used to alter a file. This will allow modification of attributes. As mentioned in the literature review, previous work on the same card was undertaken and modified the attributes of CKA\_sensitive \& CKA\_extractable from false to true. The change in this direction is forbidden by the API. However the ability for changing these at the APDU level was achieved. It still yielded no significant results, as keys and attribute are stored in separate files. And the keys could still not be loaded.

\section{C\_generateKey}
\textit{[For this trace we generate a triple DES key, key length of 24 bytes.]}

The generateKey function has 9 main components. 2 different secure messaging sessions are used to generate 24 bytes (same as the key length) on the card and send it to the API, and create the key file. The attribute file is created without the use of secure messaging as it does not contain sensitive information. The steps are listed below:\\

\begin{enumerate}
\item Open secure messaging (B.4 - commands 26, 27, 28)
\item Generate 24 random bytes and send them to the API via secure messaging (B.4 - command 29)
\item Close secure messaging (B.4 - command 30)
\item commands skipped include finding spare file for attributes and opening the file.
\item Update file with key attributes (B.4 - commands 40, 41)
\item Open secure messaging [again] (B.4 - commands 42, 43, 44)
\item Open key file directory via secure messaging (B.4 - command 45)
\item Create key file for triple DES key via secure messaging (B.4 - command 46)
\item Close secure messaging (B.4 - command 47)\\
\end{enumerate}

From this analysis of the communication trace it seems that the card is requested to generate the key value from a 'get challenge' request (within secure messaging). We assume that these same bytes are used to be stored in the key file of the triple DES key we ask it to generate. We cannot verify this at this time, as two different secure messaging sessions are used, and therefore have different session keys. This causes the encryption of the generated bytes and the storing of them in a file to be different, even if the bytes are actually used as the key value. As stated in the literature review, previous hardware security modules created by 'Athena smartcard solutions' have in the past done this without the use of secure messaging. Thus that vulnerability seems to be patched in this version of the API and smartcard, by using secure messaging when creating these bytes and storing them in the key file. 

If this assumption turns out to be correct it posses as quite a significant vulnerability that could be exploited. If the protocol for secure messaging can be reversed engineered, then we can inject our own bytes when the the request challenge is sent to the card. Or simply we can use a man in the middle attack to generate two sets of session keys. One with the attacker and the API, and one with the attacker and the smartcard. The bytes can be sent to the card in the correct format, however the attacker will also be in knowledge of the key value. This should never be the case when the attribute CKA\_Extractable is set to false.

\section{C\_generateKeyPair}
\textit{[For this trace we generate an RSA-1024 public/private key pair]}

We find 15 main components in this generation of the RSA public and private key pair. 7 components transfer data that are wrapped within an ASN.1 BER encoding [?]. These are to transfer public key information and what we believe to be file location parameters. The steps listed below give the ordering of the most important commands, and below that we give the decoding of the ASN.1 BER encoding of those 8 commands that we have found.\\

\begin{enumerate}
\item Create public key attribute file (B.5 - command 54)
\item Add public key attributes to file [inc. public exponent] (B.5 - commands 56, 57)
\item Create private key attribute file (B.5 - command 59)
\item Add private key attributes to file (B.5 - commands 61, 62)
\item Create Private CTR RSA Key file (B.5 - command 64)
\item Select temporary file (B.5 - command 65)
\item Generate RSA Key Pair (B.5 - command 66)
\item Select temporary file (B.5 - command 67)
\item Get RSA public key (B.5 - command 68)
\item Select parent folder of private key attribute file (B.5 - command 69)
\item Create new file, with public modulus and additional info (B.5 - command 70
\item Select temporary file (B.5 - command 71)
\item Get RSA public key (B.5 - command 72)
\item Select public attribute file (B.5 - command 73)
\item Add public modulus to attribute file (B.5 - command 74)
\item The rest of the communication we believe to be resetting the temporary file, and adding file location information to file control parameters of parent files. (These commands are not included in the traces within the appendix)\\\\
\end{enumerate}

The following are the ASN.1 BER decodings of the commands that have these wrapping. An online tool was used in the aid of decoding these hexadecimal bytes [?]. We provide references to the above numbering of the commands, and the appexidx references as well.\\

\textbf{1. Create public key attribute file (B.5 - command 54)}\\
Application 2(4 elem)\\
$[10]$ (1 byte) 04\\
$[03]$ (2 byte) 01 40\\
$[00]$ (2 byte) 01 A7\\
$[06]$ (8 byte) 00 20 00 20 00 20 00 20\\

\textbf{3. Create private key attribute file (B.5 - command 59)}\\
Application 2(5 elem)\\
$[10]$ (1 byte) 04\\
$[03]$ (2 byte) 02 00\\
$[00]$ (2 byte) 01 23\\
$[04]$ kx s0\\
$[06]$ (8 byte) 00 00 00 20 00 20 00 20\\

\textbf{5. Create private CTR RSA key file (B.5 - command 64)}\\
Application 2(6 elem)\\
$[10]$ (1 byte) 04\\
$[03]$ (2 byte) 00 41\\
$[00]$ (2 byte) 00 80\\
$[05]$ (5 byte) 05 0C 20 00 A3\\
$[06]$ (14 byte) 00 00 00 FF 00 FF 00 20 00 20 00 00 00 20\\
Application 17(0 elem)\\

\textbf{7. Generate RSA key pair (B.5 - command 66)}\\
$[12]$ (2 elem)\\
$[00]$ (1 byte) 06\\
$[01]$ (3 byte) 01 00 01\\

\textbf{9,13. Get RSA public key (B.5 - commands 68, 72)}\\
Application 73(2 elem)\\
$[01]$ (128 byte) D1 EF 7C A5 06 A1 87 FD 5F 13 5B 25 B7 16..\\
$[02]$ (3 byte) 01 00 01\\

\textbf{11. Create new file with public modulus and additional info (B.5 - command 70)}\\
Application 2(6 elem)\\
$[10]$ (1 byte) 04\\
$[03]$ (2 byte) 00 81\\
$[00]$ (2 byte) 00 80\\
$[05]$ (5 byte) 05 08 20 00 A3\\
$[06]$ (14 byte) 00 00 00 FF 00 FF 00 20 00 20 00 00 00 20\\
Application 17(2 elem)\\
$[16]$ (3 byte) 01 00 01\\
$[17]$ (128 byte) D1 EF 7C A5 06 A1 87 FD 5F 13 5B 25 B7...\\

RSA key pair generation is supported on the card. A dedicated processor is used for this, hence the need to change to temporary files to access the generated key and then store the public information. We see no sensitive information within the traces that are outputted in plain text. We thought that the additional information provided that is wrapped within the ASN.1 BER encoding, especially for \textbf{5}, might have exported the private key and the additional parameters required for CRT RSA keys [?]. To test this theory we deleted the generated key and generated another. The public modulus did change, however the remaining parameters did not. With a change in the public modulus, a change in the private exponent and therefore CRT parameters would also occur. Thus we concluded from that the private key is not exported out in plain text.

It seems these parameters that are listed above are used used to save the location of the RSA private key, and the public key for that matter. We have not been able to intuitively decode the proprietary encoding of file directories to locate the private key. As we will see later in the 'unwrap' function when we utilise the private key, this occurs within secure messaging and therefore cannot find the location of it from that either.

Thus without the knowledge of the proprietary encoding of file directories we do not see any vulnerabilities within the function as of yet. Studying multiple rsa key generations (without deleting the first keys) might show differences in these decodings which could lead to finding the location of the private key.


\section{C\_destroyObject}
\textit{[For this trace we delete/destroy a triple DES key and its attribute file]}

In the analysis of the destroyObject function we omitted the finding of the objects as that is the job of the findObject function and the authentication processes. However these are both crucial parts to the functions correct operation. Once the key and attribute files have been located and the user is authenticated with the card, there are 6 main components to the destruction of the object. These are listed below:\\

\begin{enumerate}
\item Select counter file for key attributes (B.6 - command 49)
\item Update counter (B.6 - command 50)
\item Select key file (B.6 - command 53)
\item Delete key file (B.6 - command 54)
\item Select key attribute file (B.6 - command 55)
\item Delete key attribute file (B.6 - command 56)
\end{enumerate}

We believe a counter file keeps track of how many keys have been created and deleted in a certain path within the cards operating system. We are unsure on its specific use as we have only seen it being updating upon deletion of an object, and the new counter value being appended to the new keys attributes (the one created after deleting an object). The original and updated counter value can be seen at B.6 - command 37 and 50 respectively.

Once this has been done the key attribute file and the key file are selected and deleted using the inter-industry command 'DELETE FILE'. This is the first time we have seen the location of a key file been accessed without secure messaging. Thus we thought this might present a vulnerability which would allow us to open and read key values at the APDU level. 

Hence we attempted to use the 'SELECT FILE' command and then 'READ BINARY' command to try and read the files data. This resulted in an error message = '69 81', meaning the command is incompatible with the file structure. With this in mind we decided to try the other command which is 'OPEN FILE CONTROL PARAMETERS' which did successfully work. The output was wrapped in an ASN.1 BER encoding, thus we decoded it which gave the following result:

Application 2(6 elem)\\
$[07]$ (1 byte) 08\\
$[03]$ (2 byte) 00 C1\\
$[00]$ (2 byte) 00 18\\
$[10]$ (1 byte) 04\\
$[06]$ (14 byte) 00 00 00 FF 00 FF 00 00 00 00 00 00 00 00\\
$[05]$ (4 byte) 01 0C 10 00\\

This seemed quite similar to what we noticed with the RSA private key analysis in the previous section. There is not present a 16 byte key that could be used for triple DES. Rather it seems that these are more pointers to more files that could hold the keys value. All security polices that we have read for 'Athena smart-card solutions' suggest that keys are held in a non-readable memory to outside actors, and only when required are encrypted internally (by the cards OS) and stored in the EEPROM for them to be used for the security operation required of them. 

Thus we did not find any security vulnerabilities from this function.

\section{C\_encrypt}
\textit{[For this function we use a triple DES key to encrypt a string 'TestString123456'] using ECB mode}


Just like 

**Authentication occurs before any security operations**
This process actually happens twice. We omit this from the traces within the appendix. The reason for it occuring twice is because the API is written in C. The data coming back from the encryption is of an unknown size. And thus the first encryption call at the APDU level is used to find the length of the data. The second call saves the data. [This seems like a bad approach to use, but does not cause security issues]

\begin{enumerate}
\item Select key file (B.7 - command 52)
\item Encrypt data (B.7 - command 53)
\end{enumerate}

\section{C\_decrypt}
% DES3-ECB decrypt

**Authentication occurs before any security operations**

This process actually happens twice. We omit this from the traces within the appendix. The reason for it occuring twice is because the API is written in C. The data coming back from the encryption is of an unknown size. And thus the first encryption call at the APDU level is used to find the length of the data. The second call saves the data. [This seems like a bad approach to use, but does not cause security issues]

There is an issue with the ECB decrypt. Removes the first 8 bytes for some reason. [this was mentioned in previous work as well.]

\begin{enumerate}
\item Select key file (B.8 - command 64)
\item Decrypt data (B.8 - command 65)
\end{enumerate}
\section{C\_setAttribute}
%refer to appendex B.9
% DES3 key label changed from 'des3' -> 'changed'
authentication and findObject steps are skipped. Both are required though.

\begin{enumerate}
\item Select key attribute file (B.9 - command 51)
\item Alter attribute file with the change in the label name (B.9 - commands 52,53)
\end{enumerate}

command chaining is used in 52 to state that 53 still needs to come in order to complete the change.


\section{C\_unwrap}
%refer to appendex B.10

\section{C\_wrap}
%refer to appendix B.11

just state that the function is still supported by the card at the APDU level.
If we can crack secure messaging, fully understand the commands for unwrap, then we can get wrap to work and find private key for RSA keys!

\begin{enumerate}
\item Wrap key (just entered the string '12345678') (B.11 - command 1)
\end{enumerate}





%---------------------------------------------------------------------------------------------------------------------------------------------------
\chapter{Attempts To Attack At the APDU Level}

In this chapter we explain the design, implementation and results of the attacks we attempted at the APDU level. The two main attacks we focus on are reverse engineering the PIN authentication protocol, and reverse engineering the secure messaging protocol. The motivations behind this come from literature stated in chapter X.

------------------------------------------------------------------------------------------------------\\
** Here give an overview of each functions possible vulnerabilities and present the motivations behind the attacks we choose. **\\
------------------------------------------------------------------------------------------------------

The smartcard we analyse uses a challenge-response algorithm to prevent the PIN being sent over in plain text. This is an improvement upon other card vendors implementations whereby they do send the PIN over in plain text for it to be verified. Successfully reverse engineering this challenge-response protocol will allow an attacker to calculate the victims PIN given one successful login trace. Furthermore, it will also remove the first dependency on the use of the API. Being able to communicate to the card directly gives an attacker a higher capability. But as of right now, since we are unaware of how the login occurs the API must be used.

The second dependancy on the use of the API, is secure messaging. When the generateKey function from the API is called to create block cipher keys, it is the computer that requests the card to generate the key, send it over using secure messaging, and then use secure messaging again to place the key into a file to be stored on the card. This again is an improvement upon previous hardware security modules from the manufacture 'Athena- Smartcard solutions' (specifically ASE KeyPro), whereby this was completed without the use of secure messaging, and therefore the block cipher keys where needlessly exported over in plain text which would allow an attacker to sniff them. If this protocol is successfully reversed engineered, it will be possible to sniff the keys by decrypting the messages which contain the keys that are in transit from computer to the smart-card.

%DO NOT FORGOT!\\
%Need to mention somewhere in this section that I have tried:\\
%delaying the response to the smart-card to test for TOTP (upto 2 hours)\\
%Modified initialization data to test if there is a master key, which could be used to create a derived key to encrypt the PIN/Password.
%(there are references to papers which suggest some smart-card manufactures do this!)

\section{Reverse Engineering PIN Authentication Protocol}

\section{Reverse Engineering PIN/password Authentication 1st draft}

The first attack that I decided to attempt is to reverse-engineer the PIN authentication method. The reasoning behind this is because if this can be successfully done, the PIN can then be inferred from one communication trace sniffed between smart-card and the API (on the computer). The inference comes from the fact that once the method is deduced, an attacker can simply brute force the possible combinations of a PIN, to test if a match is found. (This becomes clearer in the latter sections)

From previous work completed on this card by an MSC student last year [?], and from the analysis conducted in section 6.2, it was quite clear that the card has the following characteristics in terms of PIN authentication. The PKCS\#11 API requests a challenge, the smart-card responds with an 8 byte challenge, the API then calculates a 16 byte response (using the 8 byte challenge, and the PIN), the smart-card verifies whether or not the response is correct. There are two response formats to that  APDU verification command:
\begin{itemize}
\item '90 00' $\rightarrow$ verification succeeded, correct PIN was entered
\item '63 CX' $\rightarrow$ verification failed (where X is the number of attempts left before the card is blocked)\\
\end{itemize}

% IMAGE: challenge-response pin authentication
\begin{figure}[H]
\centering
\begin{subfigure}{1\textwidth}
  \includegraphics[width=1\linewidth]
  {images/section_7/7.1/authencation_method.png}
  \label{fig:sub1}
\end{subfigure}
\end{figure}


The following sections are explanations of the searches that we conducted to try and reverse-engineer the protocol showed above. To give a full understanding of how challenging this part of the project was, we will explain the combinations of possibilities we checked, and the reasoning behind each of them. These will be split up into different 'searches', and increment in terms of new findings and understanding of how the protocol may be implemented.

To aid these explanations, we introduce here 3 key sub-functionalities that most of the conducted searches use. Table 7.1 lists all the hash functions (see section 3.1) that were used, and provides the output length in bits \& bytes. Those hash functions were all supported by openSSL and the python package 'hashlib'. Table 7.2 provides the names of the bitwise logical operations that were used to 'join' two bytes together. And table 7.3 provides the description of truncation methods used to reduce the output size of a search down to 16 bytes to match the response provided by the API. \\

From here on, in the explanation of the searches I will just refer to HASH, JOIN \& TRUNCATE which will suggest that all of the elements in the tables have been iterated over and preformed on. For example TUNRCATE(HASH('this is a string')), means the string, 'this is a string', is to be hashed with all the functions in table 7.1, and then truncated to 16 bytes using all the methods listed in table 7.3.\\


%DSA & 0 & 0\\
%DSA-SHA & 0 & 0 \\
%dsaEncryption & 0 & 0 \\
%ecdsa-with-SHA1 & 0 & 0\\
% TABLE: Hash functions
\begin{table}[H]
\begin{center}
\begin{tabular}{|c|c|c|}
\hline
Hash Name & Output Length (bits) & Output Length (Bytes)\\
\hline
MD4 & 128 & 16\\
MD5 & 128 &16\\
MDC2 & 128 & 16\\
RIPEMD160 & 160 & 20\\
SHA & 160 & 20\\
SHA1 & 160 & 20\\
SHA224 & 224 & 28\\
SHA256 & 256 & 32\\
SHA384 & 384 & 48\\
SHA512 & 512 & 64\\
WHIRLPOOL & 512 & 64\\
\hline
\end{tabular}
\caption{Hash Functions}
\end{center}
\end{table}

% TABLE: Joining Methods
\begin{table}[H]
\begin{center}
\begin{tabular}{|c|}
\hline
Logical Operations\\
\hline
AND\\
OR \\
XOR\\
NOT AND\\
NOT OR\\
NOT XOR\\
\hline
\end{tabular}
\end{center}
\caption{Bitwise Logical Operations (Joins)}
\end{table}

% TABLE: Truncation methods
\begin{table}[H]
\begin{center}
\begin{tabular}{|c|p{10cm}|}
\hline
Truncation method & Description?\\
\hline
first\_16 & Truncates the output by taking the first 16 bytes \\
last\_16 & Truncates the output by taking the last 16 bytes\\
mod\_16 & Truncates the output by taking modulus $2^{128}$ [We use 128 because that's the number of bits in 16 bytes]  \\
\hline
\end{tabular}
\end{center}
\caption{Truncation Methods}
\end{table}

\newpage


\noindent Before any searches could be conducted, the first task was to extract the values of the 8 byte challenge (denoted X), and the 16 byte response (denoted Y), from a communication trace of C\_login. Table 7.4 provides the values for the PIN, X and Y in hexadecimal format.\\

 
% TABLE: PIN, X, Y
\begin{table}[H]
\begin{center}  
\begin{tabular}{|l|c|l|}
\hline
Data & ASCII & HEX\\
\hline
PIN & '0000000000000000' & 30 30 30 30 30 30 30 30 30 30 30 30 30 30 30 30\\
\hline
Y & N/A & 53 17 55 20 F4 30 18 56 80 E6 75 55 E1 91 A7 EC\\
\hline
X & N/A & 68 F1 E4 92 85 36 39 A3\\
\hline
\end{tabular}
\end{center}
\caption{}
\end{table}


In the very first original experiments, we made assumptions as did previous work on this card [ref Msc] that the PIN number was only numerical characters, only had 4-8 digits and if the PIN was less than 8 characters, it was padded using a special character to form 8 bytes. These assumptions turned out to be false, and thus the elementary experiments were all flawed from the beginning. I have not included a description of those experiments in the following sections, as many of them were in fact very similar to those explained below, just with different assumptions of the PIN. They also used a PIN for the card that was '12345', and hence there was an exponential explosion in the number of experiments for a particular search, due to needing to test for different padding schemes and characters. Hence why in these following experiments the PIN had been set to '0000000000000000' (16 zeros, no need for padding).




\subsection{Authentication Protocol Search 1.0 (Password Storage)}
Assumptions:\\
\begin{itemize}
\item PIN consists of alpha - numeric characters
\item PIN is a maximum of 16 bytes
\item PIN is encoded in ASCII characters
\item For any PIN that is less than 16 bytes long, there is padding character used to pad the PIN to 16 bytes\\
\end{itemize}

%--------------------------------------
\textbf{\underline{Search 1 - Hash functions}}\\
\noindent In this initial stages we thought that there is a large possibility that the 16 byte response was generated by hashing a combination of the PIN and the 8 byte challenge. This was partly due to common practices used in industry whereby users passwords are often hashed, and in most cases salted (see section 3.1), before storing them in databases. This practice is more secure than storing plain text passwords, as if an attacker were to gain access to the back end databases storing said passwords, the password itself would not be available to see. For authentication the password is just hashed (and salted, if a salt is used), and then compared against to the stored value. The fact that from multiple traces the 16 byte response seemed to be uniformly random, it supported this hypothesis.\\

Thus the first search that we completed focused on the fact that a hash function was used to produce the 16 byte response. Below we have listed the methods tested in expirments to generate a 16 bytes, given X and the PIN.\\

We donote $\|$ as the concatination function. Thus for two strings 'string1 ' $\|$ 'string2' = 'string1 string2'.

\textbf{\underline{Methods tested that produced a 16 byte output using X and the PIN}}
\begin{itemize}
\item join(truncate(hash(X)), pin))
\item truncate(hash(join(pin, X $\|$ X)))
\item truncate(hash(pin$\|$X))
\item truncate(hash(X$\|$pin))
\item truncate(hash(pin+X))
\item truncate(hash(join(pin, square(X))))
\end{itemize}

\textit{[The methods should be read from the most inner brackets, outward. Therefore this means that the first method dictates that X is first hashed using one of the hash functions listed in the table 7.1. The output of that is then truncated to 16 bytes using one of the functions from table 7.3. All iterations of the functions in the tables were tested.]}

The following experiment resulted in 2592 individual tests, but did not find a match to the response generated by the API [Y in table 7.4]. Thus we moved onto search 2.\\


%--------------------------------------
\textbf{\underline{Search 2 - PBKDF2}}\\
Following the failure of search 1, but still assuming there is a large possibility of a hash function being used due to the common practices mentioned in search 1, and the characteristics known so far of the 16 byte response Y. Then we decided to look at the password-based key derivation function (PBKDF2), which was created as part of PKCS \#5 by RSA laboratories [?]. It has been mentioned in literature [?], and started to be used for more secure password storage as well as for key derivation. Essentially PBKDF2 takes as input, a password (the PIN), a salt (the 8 byte challenge X), a hash function, and the number of iterative rounds. If the number of iterative round is set to 10, then the salted password would be hashed once, and the output of that would be the input for the next round of hashing. This would be completed 10 times. Literature [?] has shown that the standard for the number of iterative rounds used to be 10,000, back in 2008 (check this date). Now it is suggested to use as many rounds as is computationally feasible by the device. Due to the processing power of a smart-card I assumed that this would not exceed 100,000 rounds, in the case that PBKDF2 was used.\\

Hence this search generated experiments that ran through 1 $\rightarrow$ 100,000 rounds of PBKDF2. As the default of PBKDF2 is to truncate the output by taking the first X (X here is a variable) bytes only 'first\_16' truncation method was used.\\

\underline{\textbf{Methods tested that produced a 16 byte output using X and the PIN}}
\begin{itemize}
\item PBKDF2( hash\_function, PIN, X, number\_of\_rounds)
\end{itemize}

This generated 100,000 experiments per hash function. With 12 hash functions, this resulted in 1.2 million tests being run. Due to the sheer computational power required for this search I decided to parallelize the search based on the hash function, and run them on separate cores of a server. Even by improving the efficiency of this search, it still took 2 weeks to conduct.

Again this unfortunately did not result in a match between the 16 byte responses calculated and Y (the API's response). Hence we move onto search 3.\\

\subsection{Authentication Protocol Search 2.0 (One Time Passwords)}

%--------------------------------------
% source is OCRA!!!
\textbf{\underline{Search 3 - OCRA: OATH Challenge-Response Algorithm}}\\
With no luck of deducing the method the authentication protocol uses, we decided to look into more complex standards that exist and used in different parts of the computing industry for challenge response protocols, rather than password storage techniques. The international engineering task force (IETF) released a paper in 2011 [?]. Section 7.1 of the paper gives a one-way challenge response algorithm which fitted the characteristics of the authentication that takes between the API and the smart-card.

%Explain the one - way challenege response algorithm
Section 3.5.1 explains hash based one time passwords. But in essence HTOP is:\\
$HOTP(K,C) = Truncate(HMAC(K,C)) \& 0x7FFFFFFF$ [?-wiki ]

Still to complete.


\textbf{\underline{Search 4 - TOTP}}\\
With the above in mind, I also wanted to rule out TOTP. This was completed by halting communication between the smart-card and API using the man-in-the-middle tool (see section 4.2). I did this for upto 2 hours. We were looking for a failed verifaction, despite having the correct PIN. The failure should have been caused by the delay in the response if TOTP was used. This was not found and therefore ruled out the possibility of TOTP. 

This section will be explained better, and will also include reasoning behind why TOTP is not often used (time sync problems). \\



\subsection{Authentication Protocol Search 3.0 (Triple DES Encryption)}

Need to explain multiple logins and the characteristics found!\\
That lead me to believe DES3 encryption was used. Tabulate and cross out other possibilites!

%Note there are 4 more searches conducted that included the use of block ciphers.\\
%After that I will move onto explain how I thought there was a possibility of the use of a derived key from a globally known master key (this is due to a paper I read, on zero knowledge protocols). And hence used MiTM tool to change the serial number of the smart-card to see if an incorrect response would be produced by the API, despite having the correct PIN.\\
%
%Finally I will explain the 4 charactersistics I found by verifying with the card one after the other very quickly\\
%let Y = Y1 $\|$ Y2\\
%same PIN and X: same Y1 and Y2\\
%same PIN, different X: same Y1 different Y2\\
%different PIN, same X \& different PIN and different X: different Y1 and different Y2\\
%
%These characteristics are only present if the 'logins' happen within the same second. If the second changes then we have a different Y for the same PIN and same X. This suggests the use of a nonce (calculated by API on the computer), as well as X. My assumption is:  encrpyt(nonce$\|$X) [using a block cipher the PIN is the password]


\textbf{\underline{Search 5}}\\
Need to decrypt two different encryptions with different PINs in order to find out if this is the method used!\\
Raw ASCII password and MD5 hash (due to output size = 16 bytes)\\
DES3-ECB $\rightarrow$ encrypt ( Na $\|$ Vac)

\textbf{\underline{Search 6}}\\










\section{Block Cipher Injection (MiTM)}


\section{Manually Overriding Attributes}
This attack was found by previous literature. We simply tested to see if the same attack was feasiable on the smart card we analyse. 

To test this we created a DES3 key on the card that had the encrypt attribute set to False. We asked the card to encrypt the string '12345678'. The API returned an error stating that this function was not supported due to the attribute settings. We then override the API by manually sending the command overselves. This results in the encryption taking place on the card, when in-fact it should not. 

%---------------------------------------------------------------------------------------------------------------------------------------------------
\chapter{Conclusion / Results}

This shall summarise the whole report and my findings in regard to low-level vulnerabilities on the card. 

\chapter{Future work}

\begin{enumerate}
\item complete the reverse engineering secure messaging
\item If SM is known get wrap functionality to work from APDU layer
\item Try using 'reallocate binary' to change retry counter
\end{enumerate}

%---------------------------------------------------------------------------------------------------------------------------------------------------
%---------------------------------------------------------------------------------------------------------------------------------------------------
% use the following and \cite{} as above if you use BibTeX
% otherwise generate bibtem entries
\bibliographystyle{plain}
%\bibliography{mybib}
\begin{thebibliography}{}

%-------------------------------------------------------------------------------
% EXAMPLES!!
\bibitem{latexcompanion} 
Michel Goossens, Frank Mittelbach, and Alexander Samarin. 
\textit{The \LaTeX\ Companion}. 
Addison-Wesley, Reading, Massachusetts, 1993.
 
\bibitem{einstein} 
Albert Einstein. 
\textit{Zur Elektrodynamik bewegter K{\"o}rper}. (German) 
[\textit{On the electrodynamics of moving bodies}]. 
Annalen der Physik, 322(10):891–921, 1905.
 
\bibitem{knuthwebsite} 
Knuth: Computers and Typesetting,
\\\texttt{http://www-cs-faculty.stanford.edu/\~{}uno/abcde.html}
%------------------------------------------------------------------------------

\bibitem{virtual smartcard}
Frank Morgner : Creating a Virtual Smart Card\\
\texttt{https://frankmorgner.github.io/vsmartcard/virtualsmartcard/README.html}\\
%\href{https://frankmorgner.github.io/vsmartcard/virtualsmartcard/README.html}{link}

\bibitem{OCRA}
Internet Engineering Task Force (IETF).\\
\textit{OCRA: OATH Challenge-Response Algorithm}, June 2011\\
\texttt{https://tools.ietf.org/html/rfc6287}


\end{thebibliography}


\begin{appendices}
\chapter{Attack Traces}

\section{Multiple C\_login Traces}
\subsection{Different Pin, Different Nonce}
\begin{Verbatim}[commandchars=\\\{\}, fontsize=\small]
----- APDU command/response pair 0 -----
\textit{(Inter-Industry) Get Challenge}
00000000: 00 84 00 00 08                                    .....

00000000: \textbf{00 00 00 00 00 00 00 00}  90 00                    ..........

----- APDU command/response pair 1 -----
\textit{(Proprietary) Verify}
00000000: 80 20 00 00 \textbf{10 C2 29 5F  78 D1 68 29 13 78 BE 7B}  . ....)_x.h).x.\{
00000010: \textbf{8E 61 9B 32 2E}                                    .a.2.

00000000: 63 C9                                             c.

----- APDU command/response pair 1 -----
\textit{(Inter-Industry) Get Challenge}
00000000: 00 84 00 00 08                                    .....

00000000: \textbf{00 00 00 00 00 00 00 01}  90 00                    ..........

----- APDU command/response pair 1 -----
\textit{(Proprietary) Verify}
00000000: 80 20 00 00 10 \textbf{C4 6D 44  03 92 F3 6B EF 13 18 07}  . ....mD...k....
00000010: \textbf{CE 5A A4 B9 27}                                    .Z..'

00000000: 63 C8                                             c.
\end{Verbatim}
\subsection{Different Pin, Same Nonce}
\begin{Verbatim}[commandchars=\\\{\}, fontsize=\small]
----- APDU command/response pair 0 -----
\textit{(Inter-Industry) Get Challenge}
00000000: 00 84 00 00 08                                    .....

00000000: \textbf{00 00 00 00 00 00 00 00}  90 00                    ..........

----- APDU command/response pair 1 -----
\textit{(Proprietary) Verify}
00000000: 80 20 00 00 10 \textbf{8F 58 1B  91 BC 78 D3 37 A9 D9 FB}  . ....X...x.7...
00000010: \textbf{9C 20 58 F6 0A}                                    . X..

00000000: 63 C9                                             c.

----- APDU command/response pair 1 -----
\textit{(Inter-Industry) Get Challenge}
00000000: 00 84 00 00 08                                    .....

00000000: \textbf{00 00 00 00 00 00 00 00}  90 00                    ..........

----- APDU command/response pair 1 -----
\textit{(Proprietary) Verify}
00000000: 80 20 00 00 10 \textbf{E0 30 33  BB 03 0F 6E 11 08 C0 8D}  . ....03...n....
00000010: \textbf{1D 9D 85 C4 A6}                                    .....

00000000: 63 C8                                             c.

\end{Verbatim}
\subsection{Same Pin, Different Nonce}
\begin{Verbatim}[commandchars=\\\{\}, fontsize=\small]
----- APDU command/response pair 0 -----
\textit{(Inter-Industry) Get Challenge}
00000000: 00 84 00 00 08                                    .....

00000000: \textbf{00 00 00 00 00 00 00 00}  90 00                    ..........

----- APDU command/response pair 1 -----
\textit{(Proprietary) Verify}
00000000: 80 20 00 00 10 \textbf{6E 78 D4  D5 61 AD 3C 26 D3 89 E8}  . ...nx..a.<&...
00000010: \textbf{96 B9 92 0D 40}                                    ....@

00000000: 63 C9                                             c.

----- APDU command/response pair 1 -----
\textit{(Inter-Industry) Get Challenge}
00000000: 00 84 00 00 08                                    .....

00000000: \textbf{00 00 00 00 00 00 00 01}  90 00                    ..........

----- APDU command/response pair 1 -----
\textit{(Proprietary) Verify}
00000000: 80 20 00 00 10 \textbf{6E 78 D4  D5 61 AD 3C 26 BC C6 AA}  . ...nx..a.<&...
00000010: \textbf{72 D3 95 2B 94}                                    r..+.

00000000: 63 C8                                             c.

\end{Verbatim}
\subsection{Same Pin, Same Nonce}
\begin{Verbatim}[commandchars=\\\{\}, fontsize=\small]
----- APDU command/response pair 0 -----
\textit{(Inter-Industry) Get Challenge}
00000000: 00 84 00 00 08                                    .....

00000000: \textbf{00 00 00 00 00 00 00 00}  90 00                    ..........

----- APDU command/response pair 1 -----
\textit{(Proprietary) Verify}
00000000: 80 20 00 00 10 \textbf{EE D2 F1  54 07 18 8A 8F AB A3 F7}  . ......T.......
00000010: \textbf{3E 64 17 2D 6E}                                    >d.-n

00000000: 63 C9                                             c.

----- APDU command/response pair 1 -----
\textit{(Inter-Industry) Get Challenge}
00000000: 00 84 00 00 08                                    .....

00000000: \textbf{00 00 00 00 00 00 00 00}  90 00                    ..........

----- APDU command/response pair 1 -----
\textit{(Proprietary) Verify}
00000000: 80 20 00 00 10 \textbf{EE D2 F1  54 07 18 8A 8F AB A3 F7}  . ......T.......
00000010: \textbf{3E 64 17 2D 6E}                                    >d.-n

00000000: 63 C8                                             c.

\end{Verbatim}
\subsection{Different Second}
\begin{Verbatim}[commandchars=\\\{\}, fontsize=\small]
----- APDU command/response pair 0 -----
\textit{(Inter-Industry) Get Challenge}
00000000: 00 84 00 00 08                                    .....

00000000: \textbf{00 00 00 00 00 00 00 00}  90 00                    ..........

----- APDU command/response pair 1 -----
\textit{(Proprietary) Verify}
00000000: 80 20 00 00 10 \textbf{61 50 65  E1 AF 05 7B C3 35 98 0D}  . ...aPe...\{.5..
00000010: \textbf{DC 9D C5 42 96}                                    ...B.

00000000: 63 C9                                             c.

----- APDU command/response pair 1 -----
\textit{(Inter-Industry) Get Challenge}
00000000: 00 84 00 00 08                                    .....

00000000: \textbf{00 00 00 00 00 00 00 00}  90 00                    ..........

----- APDU command/response pair 1 -----
\textit{(Proprietary) Verify}
00000000: 80 20 00 00 10 \textbf{BF 73 83  F9 30 B1 74 D2 E4 98 83}  . ....s..0.t....
00000010: \textbf{3A 9F 1F 37 BA}                                    :..7.

00000000: 63 C8                                             c.
\end{Verbatim}

\section{Successful Login Injection}
\begin{Verbatim}[commandchars=\\\{\}, fontsize=\small]
----- APDU command/response pair 12 -----

\textit{(Inter-Industry) Get Challenge}
COMMAND from API
00000000: 00 84 00 00 08                                    .....

Do you want to automate the injection your own login response? (Y/n)

RESPONSE
00000000: E7 69 60 B5 C8 FC D2 02  90 00                    .i`.......

----- APDU command/response pair 13 -----

\textit{(Proprietary) Verify}
COMMAND from API
00000000: 80 20 00 00 10 \textbf{4A D1 3D  AB 98 7F C5 18 A9 B3 1F}  . ...J.=........
00000010: \textbf{2F 96 B4 3C AF}                                    /..<.

\textit{(Proprietary) Verify}
COMMAND injected
00000000: 80 20 00 00 10 \textbf{32 5A 9F  38 CA 4F BE 44 3A CD E1}  . ...2Z.8.O.D:..
00000010: \textbf{C5 03 84 35 DF}                                    ...5.

RESPONSE
00000000: 90 00                                             ..
\end{Verbatim}

\section{Open Secure Messaging Traces}
\subsection{Generator = 5, [Not modified]}
\begin{Verbatim}[commandchars=\\\{\}, fontsize=\small]
----- APDU command/response pair 24 -----

\textit{(Proprietary) Get Card Public Key}
COMMAND from API
00000000: 80 48 00 80 00                                    .H...

Do you want to to alter command? (y/N)

RESPONSE
00000000: \textbf{80 01 05 81 81 80} F7 B5  15 72 07 22 94 6F C4 08  .........r.".o..
00000010: 64 CB BD AF EA 55 7D BD  8F 55 36 B0 01 C2 8B 2E  d....U\}..U6.....
00000020: 32 B6 5D 45 F1 74 5D 38  12 0B AD 9D 2C 03 9C 22  2.]E.t]8....,.."
00000030: 46 68 EB 2E A2 8C 20 95  A8 2E 6C A8 E0 6D 47 F2  Fh.... ...l..mG.
00000040: D3 1E D7 01 F8 15 5C AD  DC 05 70 C0 93 B2 6D 74  ......\...p...mt
00000050: B0 9B 95 E6 4D 8C D2 FC  73 3E CD 0F 30 68 79 A5  ....M...s>..0hy.
00000060: B9 35 F2 41 3F 52 AD AD  32 A0 99 1A 18 3D CC 57  .5.A?R..2....=.W
00000070: 7E 39 DA 47 53 1E 67 15  AB 01 70 7F F2 47 96 71  ~9.GS.g...p..G.q
00000080: 44 23 CE 7B 60 67 \textbf{82 81  80} 3C 52 D2 06 89 28 92  D#.\{`g...<R...(.
00000090: 2C AB E6 3C 4E E6 DF 0E  D2 29 F1 01 BE 36 C4 F8  ,..<N....)...6..
000000A0: 54 40 56 F3 4A FA 8D 2E  9B 60 F5 07 BC ED B4 44  T@V.J....`.....D
000000B0: 56 68 5D 82 4C C4 EA D7  96 20 F8 C5 46 A6 E0 16  Vh].L.... ..F...
000000C0: B8 AB A5 D8 43 29 58 53  77 17 09 97 AA 70 68 33  ....C)XSw....ph3
000000D0: 9E F1 41 0A 5F 39 D9 75  24 7F 3A 53 63 61 47 87  ..A.\_9.u\$.:ScaG.
000000E0: 87 7F 88 96 BC BB 83 A1  CB D1 42 E0 EB 99 CF 34  ..........B....4
000000F0: 0E CA 56 4F 2C 57 50 6E  7B 1A FC 1F 90 7A E0 C2  ..VO,WPn\{....z..
00000100: 61 09                                             a.

Do you want to alter the response? (y/N)

----- APDU command/response pair 25 -----

\textit{(Inter-Industry) Get Remaining Bytes}
COMMAND from API
00000000: 00 C0 00 00 09                                    .....

Do you want to to alter command? (y/N)

RESPONSE
00000000: A8 5D D3 30 E3 5C A9 00  39 90 00                 .].0....9..

Do you want to alter the response? (y/N)

----- APDU command/response pair 26 -----

\textit{(Proprietary) Open Secure Messaging}
COMMAND from API
\textbf{00000000: 80 86 00 00 80 08 9F EA  A1 DC 8F C3 43 FD FD 4A  ............C..J}
\textbf{00000010: E6 95 7E C0 D3 C6 FE 81  61 59 4B CE 45 21 96 63  ..~.....aYK.E!.c}
\textbf{00000020: 0F AB 19 D8 61 1A B2 6B  00 E2 44 0F 06 A3 5B 60  ....a..k..D...[`}
\textbf{00000030: 87 76 C0 B7 E9 15 D5 50  DB 17 D6 C1 3C 26 54 47  .v.....P....<&TG}
\textbf{00000040: AA A3 4B DC 2C 14 81 08  84 0D F0 CA FB 49 8B C1  ..K.,........I..}
\textbf{00000050: B1 0B A1 2B 86 20 02 F2  0F 69 F0 56 2C 83 0C 6E  ...+. ...i.V,..n}
\textbf{00000060: A6 6A E9 86 56 47 71 24  0C B7 91 7F 37 85 0A D4  .j..VGq\$....7...}
\textbf{00000070: 12 35 1F CE 17 6C D2 52  FB 04 24 CF DD E9 53 BE  .5...l.R..\$...S.}
\textbf{00000080: DA 26 EA 54 FB 00                                 .&.T..}
Do you want to to alter command? (y/N)

RESPONSE
00000000: 66 56 36 31 16 42 8D 8A  BC 06 BA AC 5D 35 26 F5  fV61.B......]5&.
00000010: BF 58 15 7F 00 4F EF 2F  54 FB C4 F2 10 8F CB D6  .X...O./T.......
00000020: 90 00                                             ..

Do you want to alter the response? (y/N)

----- APDU command/response pair 27 -----

\textit{(Proprietary) Get Challenge [SM]}
COMMAND from API
00000000: 0C 84 00 00 0D 97 01 20  8E 08 05 E4 4A 19 32 DE  ....... ....J.2.
00000010: 51 CB 00                                          Q..

Do you want to to alter command? (y/N)

RESPONSE
00000000: 87 29 01 BD 69 F3 85 A7  98 2E 08 07 21 88 30 2F  .)..i.......!.0/
00000010: 06 FF 93 E4 2F 31 C5 4A  40 FB 45 3A 45 C1 4A 84  ..../1.J@.E:E.J.
00000020: 7F BA 59 BC 44 8A 70 A0  BC DA FB 99 02 90 00 8E  ..Y.D.p.........
00000030: 08 44 26 95 74 6A 51 A3  72 90 00                 .D&.tjQ.r..

Do you want to alter the response? (y/N)

----- APDU command/response pair 28 -----

\textit{(Proprietary) Close Secure Messaging}
COMMAND from API
00000000: 80 86 FF FF                                       ....
\end{Verbatim}
\subsection{Generator = 1}
\begin{Verbatim}[commandchars=\\\{\}, fontsize=\small]
----- APDU command/response pair 24 -----

\textit{(Proprietary) Get Card Public Key}
COMMAND from API
00000000: 80 48 00 80 00                                    .H...

Do you want to to alter command? (y/N)

RESPONSE
00000000: \textbf{80 01 05 81 81 80} F7 B5  15 72 07 22 94 6F C4 08  .........r.".o..
00000010: 64 CB BD AF EA 55 7D BD  8F 55 36 B0 01 C2 8B 2E  d....U\}..U6.....
00000020: 32 B6 5D 45 F1 74 5D 38  12 0B AD 9D 2C 03 9C 22  2.]E.t]8....,.."
00000030: 46 68 EB 2E A2 8C 20 95  A8 2E 6C A8 E0 6D 47 F2  Fh.... ...l..mG.
00000040: D3 1E D7 01 F8 15 5C AD  DC 05 70 C0 93 B2 6D 74  ......\...p...mt
00000050: B0 9B 95 E6 4D 8C D2 FC  73 3E CD 0F 30 68 79 A5  ....M...s>..0hy.
00000060: B9 35 F2 41 3F 52 AD AD  32 A0 99 1A 18 3D CC 57  .5.A?R..2....=.W
00000070: 7E 39 DA 47 53 1E 67 15  AB 01 70 7F F2 47 96 71  ~9.GS.g...p..G.q
00000080: 44 23 CE 7B 60 67 \textbf{82 81  80} 3C 52 D2 06 89 28 92  D#.\{`g...<R...(.
00000090: 2C AB E6 3C 4E E6 DF 0E  D2 29 F1 01 BE 36 C4 F8  ,..<N....)...6..
000000A0: 54 40 56 F3 4A FA 8D 2E  9B 60 F5 07 BC ED B4 44  T@V.J....`.....D
000000B0: 56 68 5D 82 4C C4 EA D7  96 20 F8 C5 46 A6 E0 16  Vh].L.... ..F...
000000C0: B8 AB A5 D8 43 29 58 53  77 17 09 97 AA 70 68 33  ....C)XSw....ph3
000000D0: 9E F1 41 0A 5F 39 D9 75  24 7F 3A 53 63 61 47 87  ..A.\_9.u\$.:ScaG.
000000E0: 87 7F 88 96 BC BB 83 A1  CB D1 42 E0 EB 99 CF 34  ..........B....4
000000F0: 0E CA 56 4F 2C 57 50 6E  7B 1A FC 1F 90 7A E0 C2  ..VO,WPn\{....z..
00000100: 61 09                                             a.

Do you want to alter the response? (y/N)
y

00000000: 80 01 \textbf{01} 81 81 80 F7 B5  15 72 07 22 94 6F C4 08  .........r.".o..
00000010: 64 CB BD AF EA 55 7D BD  8F 55 36 B0 01 C2 8B 2E  d....U\}..U6.....
00000020: 32 B6 5D 45 F1 74 5D 38  12 0B AD 9D 2C 03 9C 22  2.]E.t]8....,.."
00000030: 46 68 EB 2E A2 8C 20 95  A8 2E 6C A8 E0 6D 47 F2  Fh.... ...l..mG.
00000040: D3 1E D7 01 F8 15 5C AD  DC 05 70 C0 93 B2 6D 74  ......\...p...mt
00000050: B0 9B 95 E6 4D 8C D2 FC  73 3E CD 0F 30 68 79 A5  ....M...s>..0hy.
00000060: B9 35 F2 41 3F 52 AD AD  32 A0 99 1A 18 3D CC 57  .5.A?R..2....=.W
00000070: 7E 39 DA 47 53 1E 67 15  AB 01 70 7F F2 47 96 71  ~9.GS.g...p..G.q
00000080: 44 23 CE 7B 60 67 82 81  80 3C 52 D2 06 89 28 92  D#.\{`g...<R...(.
00000090: 2C AB E6 3C 4E E6 DF 0E  D2 29 F1 01 BE 36 C4 F8  ,..<N....)...6..
000000A0: 54 40 56 F3 4A FA 8D 2E  9B 60 F5 07 BC ED B4 44  T@V.J....`.....D
000000B0: 56 68 5D 82 4C C4 EA D7  96 20 F8 C5 46 A6 E0 16  Vh].L.... ..F...
000000C0: B8 AB A5 D8 43 29 58 53  77 17 09 97 AA 70 68 33  ....C)XSw....ph3
000000D0: 9E F1 41 0A 5F 39 D9 75  24 7F 3A 53 63 61 47 87  ..A.\_9.u\$.:ScaG.
000000E0: 87 7F 88 96 BC BB 83 A1  CB D1 42 E0 EB 99 CF 34  ..........B....4
000000F0: 0E CA 56 4F 2C 57 50 6E  7B 1A FC 1F 90 7A E0 C2  ..VO,WPn\{....z..
00000100: 61 09                                             a.
response changed!

----- APDU command/response pair 25 -----

\textit{(Inter-Industry) Get Remaining Bytes}
COMMAND from API
00000000: 00 C0 00 00 09                                    .....

Do you want to to alter command? (y/N)

RESPONSE
00000000: A8 5D D3 30 E3 5C A9 00  39 90 00                 .].0....9..

Do you want to alter the response? (y/N)

----- APDU command/response pair 26 -----

\textit{(Proprietary) Open Secure Messaging}
COMMAND from API
\textbf{00000000: 80 86 00 00 80 00 00 00  00 00 00 00 00 00 00 00  ................}
\textbf{00000010: 00 00 00 00 00 00 00 00  00 00 00 00 00 00 00 00  ................}
\textbf{00000020: 00 00 00 00 00 00 00 00  00 00 00 00 00 00 00 00  ................}
\textbf{00000030: 00 00 00 00 00 00 00 00  00 00 00 00 00 00 00 00  ................}
\textbf{00000040: 00 00 00 00 00 00 00 00  00 00 00 00 00 00 00 00  ................}
\textbf{00000050: 00 00 00 00 00 00 00 00  00 00 00 00 00 00 00 00  ................}
\textbf{00000060: 00 00 00 00 00 00 00 00  00 00 00 00 00 00 00 00  ................}
\textbf{00000070: 00 00 00 00 00 00 00 00  00 00 00 00 00 00 00 00  ................}
\textbf{00000080: 00 00 00 00 01 00                                 ......}

Do you want to to alter command? (y/N)

RESPONSE
00000000: 7D 2C 25 47 1C 16 34 51  E9 C3 49 38 C8 79 1E ED  \},\%G..4Q..I8.y..
00000010: A2 6B 20 D4 54 BD 67 0A  D3 85 3E B9 E0 6E D5 5E  .k .T.g...>..n.^
00000020: 90 00                                             ..

Do you want to alter the response? (y/N)

----- APDU command/response pair 27 -----

\textit{(Proprietary) Get Challenge [SM]}
COMMAND from API
00000000: 0C 84 00 00 0D 97 01 20  8E 08 08 C6 59 9B 57 E6  ....... ....Y.W.
00000010: B4 4E 00                                          .N.

Do you want to to alter command? (y/N)

RESPONSE
00000000: 69 88                                             i.

Do you want to alter the response? (y/N)

----- APDU command/response pair 28 -----

\textit{(Proprietary) Close Secure Messaging}
COMMAND from API
00000000: 80 86 FF FF                                       ....
\end{Verbatim}
\subsection{Generator = 0}
\begin{Verbatim}[commandchars=\\\{\}, fontsize=\small]
----- APDU command/response pair 24 -----

\textit{(Proprietary) Get Card Public Key}
COMMAND from API
00000000: 80 48 00 80 00                                    .H...

Do you want to to alter command? (y/N)

RESPONSE
00000000: \textbf{80 01 05 81 81 80} F7 B5  15 72 07 22 94 6F C4 08  .........r.".o..
00000010: 64 CB BD AF EA 55 7D BD  8F 55 36 B0 01 C2 8B 2E  d....U\}..U6.....
00000020: 32 B6 5D 45 F1 74 5D 38  12 0B AD 9D 2C 03 9C 22  2.]E.t]8....,.."
00000030: 46 68 EB 2E A2 8C 20 95  A8 2E 6C A8 E0 6D 47 F2  Fh.... ...l..mG.
00000040: D3 1E D7 01 F8 15 5C AD  DC 05 70 C0 93 B2 6D 74  ......\...p...mt
00000050: B0 9B 95 E6 4D 8C D2 FC  73 3E CD 0F 30 68 79 A5  ....M...s>..0hy.
00000060: B9 35 F2 41 3F 52 AD AD  32 A0 99 1A 18 3D CC 57  .5.A?R..2....=.W
00000070: 7E 39 DA 47 53 1E 67 15  AB 01 70 7F F2 47 96 71  ~9.GS.g...p..G.q
00000080: 44 23 CE 7B 60 67 \textbf{82 81  80} 3C 52 D2 06 89 28 92  D#.\{`g...<R...(.
00000090: 2C AB E6 3C 4E E6 DF 0E  D2 29 F1 01 BE 36 C4 F8  ,..<N....)...6..
000000A0: 54 40 56 F3 4A FA 8D 2E  9B 60 F5 07 BC ED B4 44  T@V.J....`.....D
000000B0: 56 68 5D 82 4C C4 EA D7  96 20 F8 C5 46 A6 E0 16  Vh].L.... ..F...
000000C0: B8 AB A5 D8 43 29 58 53  77 17 09 97 AA 70 68 33  ....C)XSw....ph3
000000D0: 9E F1 41 0A 5F 39 D9 75  24 7F 3A 53 63 61 47 87  ..A.\_9.u\$.:ScaG.
000000E0: 87 7F 88 96 BC BB 83 A1  CB D1 42 E0 EB 99 CF 34  ..........B....4
000000F0: 0E CA 56 4F 2C 57 50 6E  7B 1A FC 1F 90 7A E0 C2  ..VO,WPn\{....z..
00000100: 61 09                                             a.

Do you want to alter the response? (y/N)
y

00000000: 80 01 \textbf{00} 81 81 80 F7 B5  15 72 07 22 94 6F C4 08  .........r.".o..
00000010: 64 CB BD AF EA 55 7D BD  8F 55 36 B0 01 C2 8B 2E  d....U\}..U6.....
00000020: 32 B6 5D 45 F1 74 5D 38  12 0B AD 9D 2C 03 9C 22  2.]E.t]8....,.."
00000030: 46 68 EB 2E A2 8C 20 95  A8 2E 6C A8 E0 6D 47 F2  Fh.... ...l..mG.
00000040: D3 1E D7 01 F8 15 5C AD  DC 05 70 C0 93 B2 6D 74  ......\...p...mt
00000050: B0 9B 95 E6 4D 8C D2 FC  73 3E CD 0F 30 68 79 A5  ....M...s>..0hy.
00000060: B9 35 F2 41 3F 52 AD AD  32 A0 99 1A 18 3D CC 57  .5.A?R..2....=.W
00000070: 7E 39 DA 47 53 1E 67 15  AB 01 70 7F F2 47 96 71  ~9.GS.g...p..G.q
00000080: 44 23 CE 7B 60 67 82 81  80 3C 52 D2 06 89 28 92  D#.\{`g...<R...(.
00000090: 2C AB E6 3C 4E E6 DF 0E  D2 29 F1 01 BE 36 C4 F8  ,..<N....)...6..
000000A0: 54 40 56 F3 4A FA 8D 2E  9B 60 F5 07 BC ED B4 44  T@V.J....`.....D
000000B0: 56 68 5D 82 4C C4 EA D7  96 20 F8 C5 46 A6 E0 16  Vh].L.... ..F...
000000C0: B8 AB A5 D8 43 29 58 53  77 17 09 97 AA 70 68 33  ....C)XSw....ph3
000000D0: 9E F1 41 0A 5F 39 D9 75  24 7F 3A 53 63 61 47 87  ..A.\_9.u\$.:ScaG.
000000E0: 87 7F 88 96 BC BB 83 A1  CB D1 42 E0 EB 99 CF 34  ..........B....4
000000F0: 0E CA 56 4F 2C 57 50 6E  7B 1A FC 1F 90 7A E0 C2  ..VO,WPn\{....z..
00000100: 61 09                                             a.
response changed!

----- APDU command/response pair 25 -----

\textit{(Inter-Industry) Get Remaining Bytes}
COMMAND from API
00000000: 00 C0 00 00 09                                    .....

Do you want to to alter command? (y/N)

RESPONSE
00000000: A8 5D D3 30 E3 5C A9 00  39 90 00                 .].0....9..

Do you want to alter the response? (y/N)

----- APDU command/response pair 26 -----

\textit{(Proprietary) Open Secure Messaging}
COMMAND from API
\textbf{00000000: 80 86 00 00 80 00 00 00  00 00 00 00 00 00 00 00  ................}
\textbf{00000010: 00 00 00 00 00 00 00 00  00 00 00 00 00 00 00 00  ................}
\textbf{00000020: 00 00 00 00 00 00 00 00  00 00 00 00 00 00 00 00  ................}
\textbf{00000030: 00 00 00 00 00 00 00 00  00 00 00 00 00 00 00 00  ................}
\textbf{00000040: 00 00 00 00 00 00 00 00  00 00 00 00 00 00 00 00  ................}
\textbf{00000050: 00 00 00 00 00 00 00 00  00 00 00 00 00 00 00 00  ................}
\textbf{00000060: 00 00 00 00 00 00 00 00  00 00 00 00 00 00 00 00  ................}
\textbf{00000070: 00 00 00 00 00 00 00 00  00 00 00 00 00 00 00 00  ................}
\textbf{00000080: 00 00 00 00 00 00                                 ......}

Do you want to to alter command? (y/N)

RESPONSE
00000000: 7E 4B 40 E6 E3 B1 5D 25  2B 02 48 50 B3 63 CC 9E  ~K@...]\%+.HP.c..
00000010: 79 41 34 FC 04 B3 57 1C  06 E3 D1 36 3C 24 45 8D  yA4...W....6<\$E.
00000020: 90 00                                             ..

Do you want to alter the response? (y/N)

----- APDU command/response pair 27 -----

\textit{(Proprietary) Get Challenge [SM]}
COMMAND from API
00000000: 0C 84 00 00 0D 97 01 20  8E 08 99 BD 52 69 31 DD  ....... ....Ri1.
00000010: DB FD 00                                          ...

Do you want to to alter command? (y/N)

RESPONSE
00000000: 69 88                                             i.

Do you want to alter the response? (y/N)

----- APDU command/response pair 28 -----

\textit{(Proprietary) Close Secure Messaging}
COMMAND from API
00000000: 80 86 FF FF                                       ....
\end{Verbatim}


\section{Overriding Attribute Controls}
\section{Encrypt\_False}


\chapter{API Function Traces}

\section{Initialization}
\begin{Verbatim}[commandchars=\\\{\}, fontsize=\small]
----- APDU command/response pair 1 -----
00000000: 00 A4 04 00 0C A0 00 00  01 64 4C 41 53 45 52 00  .........dLASER.
00000010: 01 00                                             ..

00000000: 90 00                                             ..

----- APDU command/response pair 4 -----
00000000: 80 A4 08 00 06 3F 00 30  00 C0 00                 .....?.0...

00000000: 90 00                                             ..


----- APDU command/response pair 5 -----
00000000: 00 B0 00 00 00                                    .....

00000000: 49 44 50 72 6F 74 65 63  74 20 20 20 20 20 20 20  IDProtect       
00000010: 20 20 20 20 20 20 20 20  20 20 20 20 20 20 20 20                  
00000020: 41 74 68 65 6E 61 20 53  6D 61 72 74 63 61 72 64  Athena Smartcard
00000030: 20 53 6F 6C 75 74 69 6F  6E 73 20 20 20 20 20 20   Solutions      
00000040: 49 44 50 72 6F 74 65 63  74 20 20 20 20 20 20 20  IDProtect       
00000050: 30 44 35 30 30 30 30 39  32 31 32 32 38 37 39 36  0D50000921228796
00000060: 0D 04 00 00 00 00 00 00  00 00 00 00 00 00 00 00  ................
00000070: 00 00 00 00 10 00 00 00  04 00 00 00 FF FF FF FF  ................
00000080: 00 00 00 00 FF FF FF FF  00 00 00 00 01 00 01 00  ................
00000090: 00 00 00 00 00 00 00 00  00 00 00 00 00 00 00 00  ................
000000A0: 00 90 00                                          ...


----- APDU command/response pair 8 -----
00000000: 80 A4 08 00 08 3F 00 30  00 30 03 40 00           .....?.0.0.@.

00000000: 90 00                                             ..


----- APDU command/response pair 9 -----
00000000: 00 B0 00 02 64                                    ....d

00000000: 41 54 48 45 4E 41 53 4E  C0 AD AA 78 FC 88 42 0D  ATHENASN...x..B.
00000010: 90 00                                             ..

\end{Verbatim}

\section{C\_login}
\begin{Verbatim}[commandchars=\\\{\}, fontsize=\small]


----- APDU command/response pair 10 -----
00000000: 80 A4 08 0C 04 \textbf{3F 00 00  20 00}                    .....?.. .

00000000: 62 2F 87 01 08 83 02 00  20 80 02 00 10 8A 01 04  b/...... .......
00000010: 86 0E 00 FF C0 30 00 FF  00 10 00 FF 00 10 00 00  .....0..........
00000020: 85 0F 00 01 00 00 \textbf{AA} 00  04 10 00 00 00 00 00 FF  ................
00000030: FF 90 00                                          ...

----- APDU command/response pair 11 -----
00000000: 80 A4 08 00 04 \textbf{3F 00 00  20}                       .....?.. 

00000000: 90 00                                             ..

----- APDU command/response pair 12 -----
00000000: 00 84 00 00 08                                    .....

00000000: \textbf{11 B7 B2 80 4B 17 0D A4}  90 00                    ....K.....


----- APDU command/response pair 13 -----
00000000: 80 20 00 00 10 \textbf{1D ED 9E  47 A8 C9 EA CE 37 82 2C}  . ......G....7.,
00000010: \textbf{92 CF 07 20 2D}                                    ... -

00000000: 90 00                                             ..


----- APDU command/response pair 20 -----
00000000: 80 28 00 00 04 00 00 00  20                       .(...... 

00000000: 90 00                                             ..

\end{Verbatim}
%\pagebreak

\section{C\_findObject}
\begin{Verbatim}[commandchars=\\\{\}, fontsize=\small]
----- APDU command/response pair 32 -----
00000000: 80 30 01 00 00                                    .0...

00000000: D1 02 00 03 D2 02 03 40  D2 02 03 46 D2 0A 86 7F  .......@...F....
00000010: 63 6D 61 70 66 69 6C 65  90 00                    cmapfile..


----- APDU command/response pair 33 -----
00000000: 80 A4 08 00 06 3F 00 30  00 30 02                 .....?.0.0.

00000000: 90 00                                             ..


----- APDU command/response pair 34 -----
00000000: 80 30 01 00 00                                    .0...

00000000: D1 02 00 00 90 00                                 ......


----- APDU command/response pair 35 -----
00000000: 80 A4 08 00 08 3F 00 30  00 30 01 03 40           .....?.0.0..@

00000000: 90 00                                             ..


----- APDU command/response pair 36 -----
00000000: 00 B0 00 00 00                                    .....

00000000: 00 03 03 40 01 23 18 00  00 00 00 04 04 00 00 00  ...@.\#..........
00000010: 00 01 00 00 01 01 00 02  00 00 01 00 00 03 10 00  ................
00000020: 04 64 65 73 33 FF FF FF  FF FF FF FF FF FF FF FF  .des3...........
00000030: FF FF FF FF FF FF FF FF  FF FF FF FF FF FF FF FF  ................
00000040: FF FF FF FF FF FF FF FF  FF FF FF FF FF FF FF FF  ................
00000050: FF FF FF FF FF FF FF FF  FF FF FF FF FF FF FF FF  ................
00000060: FF 00 11 01 00 18 FF FF  FF FF FF FF FF FF FF FF  ................
00000070: FF FF FF FF FF FF FF FF  FF FF FF FF FF FF 01 00  ................
00000080: 00 00 04 15 00 00 00 01  02 10 00 01 01 FF FF FF  ................
00000090: FF FF FF FF FF FF FF FF  FF FF FF FF FF FF FF FF  ................
000000A0: FF FF FF FF FF FF FF FF  FF FF FF FF 01 03 30 00  ..............0.
000000B0: 01 01 01 04 00 00 01 01  01 05 50 00 01 01 01 06  ..........P.....
000000C0: 00 00 01 00 01 07 50 00  01 01 01 08 50 00 01 01  ......P.....P...
000000D0: 01 0A 00 00 01 01 01 0C  10 00 01 00 01 10 10 00  ................
000000E0: 00 FF FF FF FF FF FF FF  FF 01 11 10 00 00 FF FF  ................
000000F0: FF FF FF FF FF FF 01 62  50 00 01 00 01 63 00 00  .......bP....c..
00000100: 61 27                                             a'


----- APDU command/response pair 37 -----
00000000: 00 B0 01 00 00                                    .....

00000000: 01 01 01 64 00 00 01 01  01 65 00 00 01 01 01 66  ...d.....e.....f
00000010: 00 00 04 31 01 00 00 01  70 00 00 01 01 80 10 00  ...1....p.......
00000020: 00 01 00 99 03 99 03 90  00                       .........


----- APDU command/response pair 38 -----
00000000: 80 A4 08 00 08 3F 00 30  00 30 01 03 46           .....?.0.0..F

00000000: 90 00      

----- APDU command/response pair 39 -----
00000000: 00 B0 00 00 00                                    .....

00000000: 00 00 00 00 00 00 00 00  00 00 00 00 00 00 00 00  ................
00000010: 00 00 00 00 00 00 00 00  00 00 00 00 00 00 00 00  ................
00000020: 00 00 00 00 00 00 00 00  00 00 00 00 00 00 00 00  ................
00000030: 00 00 00 00 00 00 00 00  00 00 00 00 00 00 00 00  ................
00000040: 00 00 00 00 00 00 00 00  00 00 00 00 00 00 00 00  ................
00000050: 00 00 00 00 00 00 00 00  00 00 00 00 00 00 00 00  ................
00000060: 00 00 00 00 00 00 00 00  00 00 00 00 00 00 00 00  ................
00000070: 00 00 00 00 00 00 00 00  00 00 00 00 00 00 00 00  ................
00000080: 00 00 00 00 00 00 00 00  00 00 00 00 00 00 00 00  ................
00000090: 00 00 00 00 00 00 00 00  00 00 00 00 00 00 00 00  ................
000000A0: 00 00 00 00 00 00 00 00  00 00 00 00 00 00 00 00  ................
000000B0: 00 00 00 00 00 00 00 00  00 00 00 00 00 00 00 00  ................
000000C0: 00 00 00 00 00 00 00 00  00 00 00 00 00 00 00 00  ................
000000D0: 00 00 00 00 00 00 00 00  00 00 00 00 00 00 00 00  ................
000000E0: 00 00 00 00 00 00 00 00  00 00 00 00 00 00 00 00  ................
000000F0: 00 00 00 00 00 00 00 00  00 00 00 00 00 00 00 00  ................
00000100: 61 2F                                             a/


----- APDU command/response pair 40 -----
00000000: 00 B0 01 00 00                                    .....

00000000: 00 00 00 00 00 00 00 00  00 00 00 00 00 00 00 00  ................
00000010: 00 00 00 00 00 00 00 00  00 00 00 00 00 00 00 00  ................
00000020: 00 00 00 00 00 00 00 00  00 00 00 00 00 00 00 90  ................
00000030: 00                                                .                                       ..
\end{Verbatim}
\pagebreak

\section{C\_generateKey}
\begin{Verbatim}[commandchars=\\\{\}, fontsize=\small]
----- APDU command/response pair 26 -----
00000000: 80 48 00 80 00                                    .H...

00000000: 80 01 05 81 81 80 F7 B5  15 72 07 22 94 6F C4 08  .........r.".o..
00000010: 64 CB BD AF EA 55 7D BD  8F 55 36 B0 01 C2 8B 2E  d....U\}..U6.....
00000020: 32 B6 5D 45 F1 74 5D 38  12 0B AD 9D 2C 03 9C 22  2.]E.t]8....,.."
00000030: 46 68 EB 2E A2 8C 20 95  A8 2E 6C A8 E0 6D 47 F2  Fh.... ...l..mG.
00000040: D3 1E D7 01 F8 15 5C AD  DC 05 70 C0 93 B2 6D 74  ..........p...mt
00000050: B0 9B 95 E6 4D 8C D2 FC  73 3E CD 0F 30 68 79 A5  ....M...s>..0hy.
00000060: B9 35 F2 41 3F 52 AD AD  32 A0 99 1A 18 3D CC 57  .5.A?R..2....=.W
00000070: 7E 39 DA 47 53 1E 67 15  AB 01 70 7F F2 47 96 71  ~9.GS.g...p..G.q
00000080: 44 23 CE 7B 60 67 82 81  80 3C 52 D2 06 89 28 92  D#.\{`g...<R...(.
00000090: 2C AB E6 3C 4E E6 DF 0E  D2 29 F1 01 BE 36 C4 F8  ,..<N....)...6..
000000A0: 54 40 56 F3 4A FA 8D 2E  9B 60 F5 07 BC ED B4 44  T@V.J....`.....D
000000B0: 56 68 5D 82 4C C4 EA D7  96 20 F8 C5 46 A6 E0 16  Vh].L.... ..F...
000000C0: B8 AB A5 D8 43 29 58 53  77 17 09 97 AA 70 68 33  ....C)XSw....ph3
000000D0: 9E F1 41 0A 5F 39 D9 75  24 7F 3A 53 63 61 47 87  ..A.\_9.u\$.:ScaG.
000000E0: 87 7F 88 96 BC BB 83 A1  CB D1 42 E0 EB 99 CF 34  ..........B....4
000000F0: 0E CA 56 4F 2C 57 50 6E  7B 1A FC 1F 90 7A E0 C2  ..VO,WPn\{....z..
00000100: 61 09                                             a.


----- APDU command/response pair 27 -----
00000000: 00 C0 00 00 09                                    .....

00000000: A8 5D D3 30 E3 5C A9 00  39 90 00                 .].0....9..


----- APDU command/response pair 28 -----
00000000: 80 86 00 00 80 84 7F A0  E7 6C 8F AA 50 9C C3 6E  .........l..P..n
00000010: 82 5E 84 B6 E4 F6 77 1C  45 FA AB 06 1B 24 C4 A8  .^....w.E....\$..
00000020: 92 03 A9 9C A8 2B BE 1B  28 C4 57 83 A5 5E BB 8D  .....+..(.W..^..
00000030: D2 BF 3F D5 02 8A 7C 13  10 9C 75 06 91 1A 0F 05  ..?...|...u.....
00000040: 55 B4 C9 12 8A 69 59 B6  07 1D 67 F2 8A C9 FA BC  U....iY...g.....
00000050: F3 BE 16 73 51 C0 76 0C  11 E5 0C D3 8C FE 09 E5  ...sQ.v.........
00000060: 1E 52 DE 38 D9 AC 2D EB  C6 A1 C4 8E ED 03 7D 07  .R.8..-.......\}.
00000070: 85 B7 FE 66 82 2F 03 65  94 DC 27 77 2B 3A 28 71  ...f./.e..'w+:(q
00000080: 97 08 5D 03 80 00                                 ..]...

00000000: F9 D0 66 F7 48 CB BB E8  CE 93 60 05 99 1B 81 2E  ..f.H.....`.....
00000010: 73 0B B7 B8 DC 10 A7 84  B3 99 D8 C8 60 D6 48 5A  s...........`.HZ
00000020: 90 00                                             ..


----- APDU command/response pair 29 -----
00000000: 0C 84 00 00 0D 97 01 18  8E 08 2B 88 7C 0C 8C 24  ..........+.|..\$
00000010: 00 1F 00                                          ...

00000000: 87 21 01 69 AB B7 01 F5  F5 8E EA B8 F3 09 D7 5E  .!.i...........^
00000010: F5 26 3C 7F 1D 15 90 B8  40 D4 A1 85 9C 57 3F 27  .&<.....@....W?'
00000020: 87 84 C6 99 02 90 00 8E  08 42 84 88 19 99 3B C2  .........B....;.
00000030: 10 90 00                                          ...


----- APDU command/response pair 30 -----
00000000: 80 86 FF FF                                       ....

00000000: 90 00                                             ..


----- APDU command/response pair 39 -----
00000000: 80 A4 08 00 08 3F 00 30  00 30 01 03 40           .....?.0.0..@

00000000: 90 00                                             ..


----- APDU command/response pair 40 -----
00000000: 00 D6 00 00 FA 01 03 03  40 01 23 18 00 00 00 00  ........@.#.....
00000010: 04 04 00 00 00 00 01 00  00 01 01 00 02 00 00 01  ................
00000020: 00 00 03 10 00 04 64 65  73 33 FF FF FF FF FF FF  ......des3......
00000030: FF FF FF FF FF FF FF FF  FF FF FF FF FF FF FF FF  ................
00000040: FF FF FF FF FF FF FF FF  FF FF FF FF FF FF FF FF  ................
00000050: FF FF FF FF FF FF FF FF  FF FF FF FF FF FF FF FF  ................
00000060: FF FF FF FF FF FF 00 11  01 00 18 FF FF FF FF FF  ................
00000070: FF FF FF FF FF FF FF FF  FF FF FF FF FF FF FF FF  ................
00000080: FF FF FF 01 00 00 00 04  15 00 00 00 01 02 10 00  ................
00000090: 01 01 FF FF FF FF FF FF  FF FF FF FF FF FF FF FF  ................
000000A0: FF FF FF FF FF FF FF FF  FF FF FF FF FF FF FF FF  ................
000000B0: FF 01 03 30 00 01 01 01  04 00 00 01 01 01 05 50  ...0...........P
000000C0: 00 01 01 01 06 00 00 01  00 01 07 50 00 01 01 01  ...........P....
000000D0: 08 50 00 01 01 01 0A 00  00 01 01 01 0C 10 00 01  .P..............
000000E0: 00 01 10 10 00 00 FF FF  FF FF FF FF FF FF 01 11  ................
000000F0: 10 00 00 FF FF FF FF FF  FF FF FF 01 62 50 00     ............bP.

00000000: 90 00                                             ..


----- APDU command/response pair 41 -----
00000000: 00 D6 00 FA 2D 01 00 01  63 00 00 01 01 01 64 00  ....-...c.....d.
00000010: 00 01 01 01 65 00 00 01  01 01 66 00 00 04 31 01  ....e.....f...1.
00000020: 00 00 01 70 00 00 01 01  80 10 00 00 01 00 88 03  ...p............
00000030: 88 03                                             ..

00000000: 90 00                                             ..


----- APDU command/response pair 42 -----
00000000: 80 48 00 80 00                                    .H...

00000000: 80 01 05 81 81 80 F7 B5  15 72 07 22 94 6F C4 08  .........r.".o..
00000010: 64 CB BD AF EA 55 7D BD  8F 55 36 B0 01 C2 8B 2E  d....U\}..U6.....
00000020: 32 B6 5D 45 F1 74 5D 38  12 0B AD 9D 2C 03 9C 22  2.]E.t]8....,.."
00000030: 46 68 EB 2E A2 8C 20 95  A8 2E 6C A8 E0 6D 47 F2  Fh.... ...l..mG.
00000040: D3 1E D7 01 F8 15 5C AD  DC 05 70 C0 93 B2 6D 74  ......\...p...mt
00000050: B0 9B 95 E6 4D 8C D2 FC  73 3E CD 0F 30 68 79 A5  ....M...s>..0hy.
00000060: B9 35 F2 41 3F 52 AD AD  32 A0 99 1A 18 3D CC 57  .5.A?R..2....=.W
00000070: 7E 39 DA 47 53 1E 67 15  AB 01 70 7F F2 47 96 71  ~9.GS.g...p..G.q
00000080: 44 23 CE 7B 60 67 82 81  80 3C 52 D2 06 89 28 92  D#.\{`g...<R...(.
00000090: 2C AB E6 3C 4E E6 DF 0E  D2 29 F1 01 BE 36 C4 F8  ,..<N....)...6..
000000A0: 54 40 56 F3 4A FA 8D 2E  9B 60 F5 07 BC ED B4 44  T@V.J....`.....D
000000B0: 56 68 5D 82 4C C4 EA D7  96 20 F8 C5 46 A6 E0 16  Vh].L.... ..F...
000000C0: B8 AB A5 D8 43 29 58 53  77 17 09 97 AA 70 68 33  ....C)XSw....ph3
000000D0: 9E F1 41 0A 5F 39 D9 75  24 7F 3A 53 63 61 47 87  ..A.\_9.u\$.:ScaG.
000000E0: 87 7F 88 96 BC BB 83 A1  CB D1 42 E0 EB 99 CF 34  ..........B....4
000000F0: 0E CA 56 4F 2C 57 50 6E  7B 1A FC 1F 90 7A E0 C2  ..VO,WPn\{....z..
00000100: 61 09                                             a.


----- APDU command/response pair 43 -----
00000000: 00 C0 00 00 09                                    .....

00000000: A8 5D D3 30 E3 5C A9 00  39 90 00                 .].0.\..9..


----- APDU command/response pair 44 -----
00000000: 80 86 00 00 80 C3 88 FD  AF 64 0D 35 77 85 D4 20  .........d.5w.. 
00000010: 57 10 02 F4 1E 38 51 37  40 31 7F 7F 11 E8 4B 8D  W....8Q7@1....K.
00000020: A5 CE C0 50 EB 6B CE E6  E0 DE E8 34 7C FE 0B 6C  ...P.k.....4|..l
00000030: F0 70 9F E3 5D F7 AA 50  BB 1C F6 8C 00 1B 18 EA  .p..]..P........
00000040: BF 73 E4 BE 75 B6 AE 29  B1 A2 A3 B8 1D 52 FD 19  .s..u..).....R..
00000050: C9 CA 20 FB 80 C2 20 A9  E3 A6 15 6C 11 B3 E9 18  .. ... ....l....
00000060: 13 3F 65 02 28 21 74 72  29 EA E2 27 8B DA 3E 45  .?e.(!tr)..'..>E
00000070: 82 A1 B0 D9 A7 1A 3D F3  5D 4D 27 F4 D2 73 ED 0F  ......=.]M'..s..
00000080: A8 88 41 F2 4F 00                                 ..A.O.

00000000: 14 8C 30 9E D5 10 25 B1  F7 AF 07 E7 25 8B 22 3C  ..0...\%.....\%."<
00000010: 62 61 8F 24 FB 59 E1 63  D7 B1 08 6D 07 7A DD 93  ba.\$.Y.c...m.z..
00000020: 90 00                                             ..


----- APDU command/response pair 45 -----
00000000: 8C A4 08 00 15 87 09 01  E5 61 A8 BF 89 AD D7 FF  .........a......
00000010: 8E 08 C2 B3 32 7B D7 83  C9 D1                    ....2\{....

00000000: 99 02 90 00 8E 08 E6 37  E6 BE 12 F8 73 6F 90 00  .......7....so..


----- APDU command/response pair 46 -----
00000000: 0C E0 08 00 4D 87 41 01  41 03 69 5A A4 EE 5F 44  ....M.A.A.iZ.._D
00000010: 2C 4C A9 FE 46 8D 1F 5B  79 D6 89 68 EB 94 CF FB  ,L..F..[y..h....
00000020: 6B A2 55 F6 65 B7 19 66  B3 67 E0 DF 46 F2 27 22  k.U.e..f.g..F.'"
00000030: AC D8 C1 57 C5 54 5B DF  B9 10 87 58 81 2E 9E 65  ...W.T[....X...e
00000040: 07 B1 6E 14 F8 DE 09 AF  8E 08 8C 79 AD C4 3B E2  ..n........y..;.
00000050: D2 84                                             ..


00000000: 99 02 90 00 8E 08 A5 D0  49 2A C0 91 47 68 90 00  ........I*..Gh..

----- APDU command/response pair 47 -----
00000000: 80 86 FF FF                                       ....

00000000: 90 00                                             ..
\end{Verbatim}
%\pagebreak

\section{C\_generateKeyPair}
\begin{Verbatim}[commandchars=\\\{\}, fontsize=\small]
----- APDU command/response pair 54 -----
00000000: 00 E0 01 00 18 62 81 15  8A 01 04 83 02 01 40 80  .....b........@.
00000010: 02 01 A7 86 08 00 20 00  20 00 20 00 20           ...... . . . 

00000000: 90 00                                             ..


----- APDU command/response pair 55 -----
00000000: 80 A4 08 00 08 3F 00 30  00 30 02 01 40           .....?.0.0..@

00000000: 90 00                                             ..


----- APDU command/response pair 56 -----
00000000: 00 D6 00 00 FA 01 01 01  40 01 A3 16 00 00 00 00  ........@.......
00000010: 04 02 00 00 00 00 01 00  00 01 01 00 02 00 00 01  ................
00000020: 01 00 03 10 00 03 70 75  62 FF FF FF FF FF FF FF  ......pub.......
00000030: FF FF FF FF FF FF FF FF  FF FF FF FF FF FF FF FF  ................
00000040: FF FF FF FF FF FF FF FF  FF FF FF FF FF FF FF FF  ................
00000050: FF FF FF FF FF FF FF FF  FF FF FF FF FF FF FF FF  ................
00000060: FF FF FF FF FF FF 00 86  32 00 01 00 01 00 00 00  ........2.......
00000070: 04 00 00 00 00 01 01 10  00 00 FF FF FF FF FF FF  ................
00000080: FF FF FF FF FF FF FF FF  FF FF FF FF FF FF FF FF  ................
00000090: FF FF FF FF FF FF FF FF  FF FF 01 02 10 00 01 03  ................
000000A0: FF FF FF FF FF FF FF FF  FF FF FF FF FF FF FF FF  ................
000000B0: FF FF FF FF FF FF FF FF  FF FF FF FF FF FF FF 01  ................
000000C0: 04 00 00 01 01 01 06 00  00 01 01 01 0A 00 00 01  ................
000000D0: 01 01 0B 00 00 01 00 01  0C 10 00 01 00 01 10 10  ................
000000E0: 00 00 FF FF FF FF FF FF  FF FF 01 11 10 00 00 FF  ................
000000F0: FF FF FF FF FF FF FF 01  20 00 00 80 A8 FD 0C     ........ ......

00000000: 90 00                                             ..


----- APDU command/response pair 57 -----
00000000: 00 D6 00 FA AD 53 6B 7F  00 00 A8 FD 0C 53 6B 7F  .....Sk......Sk.
00000010: 00 00 B0 AA 47 51 6B 7F  00 00 00 30 00 00 00 00  ....GQk....0....
00000020: 00 00 B0 AA 47 51 6B 7F  00 00 02 30 00 00 00 00  ....GQk....0....
00000030: 00 00 00 00 00 00 00 00  00 00 00 00 00 00 00 00  ................
00000040: 00 00 00 00 00 00 00 00  00 00 11 01 00 00 00 00  ................
00000050: 00 00 58 FD 0C 53 6B 7F  00 00 58 FD 0C 53 6B 7F  ..X..Sk...X..Sk.
00000060: 00 00 00 00 00 00 00 00  00 00 00 00 00 00 00 00  ................
00000070: 00 00 00 00 00 00 00 00  00 00 00 00 00 00 00 00  ................
00000080: 00 00 01 21 00 00 04 00  04 00 00 01 22 00 00 03  ...!........"...
00000090: 01 00 01 01 63 00 00 01  01 01 66 00 00 04 00 00  ....c.....f.....
000000A0: 00 00 01 70 00 00 01 01  80 10 00 00 01 00 93 03  ...p............
000000B0: 93 03                                             ..

00000000: 90 00                                             ..


----- APDU command/response pair 58 -----
00000000: 80 A4 08 00 06 3F 00 30  00 30 02                 .....?.0.0.

00000000: 90 00                                             ..


----- APDU command/response pair 59 -----
00000000: 00 E0 01 00 1E 62 81 1B  8A 01 04 83 02 02 00 80  .....b..........
00000010: 02 01 23 84 04 6B 78 73  30 86 08 00 00 00 20 00  ..\#..kxs0..... .
00000020: 20 00 20                                           . 

00000000: 90 00                                             ..


----- APDU command/response pair 60 -----
00000000: 80 A4 08 00 08 3F 00 30  00 30 02 02 00           .....?.0.0...

00000000: 90 00                                             ..


----- APDU command/response pair 61 -----
00000000: 00 D6 00 00 FA 01 02 02  00 01 1F 16 00 00 00 00  ................
00000010: 04 03 00 00 00 00 01 00  00 01 01 00 02 00 00 01  ................
00000020: 01 00 03 10 00 04 70 72  69 76 FF FF FF FF FF FF  ......priv......
00000030: FF FF FF FF FF FF FF FF  FF FF FF FF FF FF FF FF  ................
00000040: FF FF FF FF FF FF FF FF  FF FF FF FF FF FF FF FF  ................
00000050: FF FF FF FF FF FF FF FF  FF FF FF FF FF FF FF FF  ................
00000060: FF FF FF FF FF FF 01 00  00 00 04 00 00 00 00 01  ................
00000070: 01 10 00 00 FF FF FF FF  FF FF FF FF FF FF FF FF  ................
00000080: FF FF FF FF FF FF FF FF  FF FF FF FF FF FF FF FF  ................
00000090: FF FF FF FF 01 02 10 00  01 03 FF FF FF FF FF FF  ................
000000A0: FF FF FF FF FF FF FF FF  FF FF FF FF FF FF FF FF  ................
000000B0: FF FF FF FF FF FF FF FF  FF 01 03 30 00 01 01 01  ...........0....
000000C0: 05 50 00 01 01 01 07 50  00 01 01 01 08 50 00 01  .P.....P.....P..
000000D0: 01 01 09 50 00 01 00 01  0C 10 00 01 00 01 10 10  ...P............
000000E0: 00 00 FF FF FF FF FF FF  FF FF 01 11 10 00 00 FF  ................
000000F0: FF FF FF FF FF FF FF 01  62 50 00 01 01 01 63     ........bP....c

00000000: 90 00                                             ..


----- APDU command/response pair 62 -----
00000000: 00 D6 00 FA 29 00 00 01  01 01 64 00 00 01 00 01  ....).....d.....
00000010: 65 00 00 01 01 01 66 00  00 04 00 00 00 00 01 70  e.....f........p
00000020: 00 00 01 01 80 10 00 00  01 00 93 03 93 03        ..............

00000000: 90 00                                             ..


----- APDU command/response pair 63 -----
00000000: 80 A4 08 00 06 3F 00 30  00 30 02                 .....?.0.0.

00000000: 90 00                                             ..


----- APDU command/response pair 64 -----
00000000: 00 E0 08 00 27 62 81 24  8A 01 04 83 02 00 41 80  ....'b.\$......A.
00000010: 02 00 80 85 05 05 0C 20  00 A3 86 0E 00 00 00 FF  ....... ........
00000020: 00 FF 00 20 00 20 00 00  00 20 71 00              ... . ... q.

00000000: 90 00                                             ..


----- APDU command/response pair 65 -----
00000000: 80 A4 00 00 02 00 41                              ......A

00000000: 90 00                                             ..


----- APDU command/response pair 66 -----
00000000: 00 47 00 00 0C AC 81 09  80 01 06 81 81 03 01 00  .G..............
00000010: 01                                                .

00000000: 90 00                                             ..


----- APDU command/response pair 67 -----
00000000: 80 A4 08 00 08 3F 00 30  00 30 02 00 41           .....?.0.0..A

00000000: 90 00                                             ..


----- APDU command/response pair 68 -----
00000000: 80 48 00 00 00                                    .H...

00000000: \textbf{7F 49 81 88 81} \textbf{81 80} D1  EF 7C A5 06 A1 87 FD 5F  .I.......|.....\_
00000010: 13 5B 25 B7 16 B9 BA A7  21 43 3D DB 51 9D C1 D1  .[\%.....!C=.Q...
00000020: 5A 3C 95 7C B6 F0 37 57  83 CF 2D 0B 53 66 C7 11  Z<.|..7W..-.Sf..
00000030: D5 6B FD 28 FA A0 EA 50  1E 2B FD B5 09 49 E2 E7  .k.(...P.+...I..
00000040: 51 67 1B 00 B0 9D 52 CD  22 D8 69 8C 36 74 54 41  Qg....R.".i.6tTA
00000050: 6E 40 58 4F 79 52 E4 D9  00 43 9C 2C 79 FE A6 48  n@XOyR...C.,y..H
00000060: B7 31 8A B2 05 04 C4 DD  B3 86 E6 4F 38 A6 5D 2A  .1.........O8.]*
00000070: CD A8 3F 95 E4 FF 7B 05  1E ED 4A B5 99 69 36 F0  ..?...\{...J..i6.
00000080: B9 5B 29 C6 EC B3 25 \textbf{82  03} 01 00 01 90 00        .[)...\%.......


----- APDU command/response pair 69 -----
00000000: 80 A4 08 00 06 3F 00 30  00 30 02                 .....?.0.0.

00000000: 90 00                                             ..


----- APDU command/response pair 70 -----
00000000: 00 E0 08 00 B0 62 81 AD  8A 01 04 83 02 00 81 80  .....b..........
00000010: 02 00 80 85 05 05 08 20  00 A3 86 0E 00 00 00 FF  ....... ........
00000020: 00 FF 00 20 00 20 00 00  00 20 71 81 88 90 03 01  ... . ... q.....
00000030: 00 01 91 81 80 D1 EF 7C  A5 06 A1 87 FD 5F 13 5B  .......|.....\_.[
00000040: 25 B7 16 B9 BA A7 21 43  3D DB 51 9D C1 D1 5A 3C  \%.....!C=.Q...Z<
00000050: 95 7C B6 F0 37 57 83 CF  2D 0B 53 66 C7 11 D5 6B  .|..7W..-.Sf...k
00000060: FD 28 FA A0 EA 50 1E 2B  FD B5 09 49 E2 E7 51 67  .(...P.+...I..Qg
00000070: 1B 00 B0 9D 52 CD 22 D8  69 8C 36 74 54 41 6E 40  ....R.".i.6tTAn@
00000080: 58 4F 79 52 E4 D9 00 43  9C 2C 79 FE A6 48 B7 31  XOyR...C.,y..H.1
00000090: 8A B2 05 04 C4 DD B3 86  E6 4F 38 A6 5D 2A CD A8  .........O8.]*..
000000A0: 3F 95 E4 FF 7B 05 1E ED  4A B5 99 69 36 F0 B9 5B  ?...\{...J..i6..[
000000B0: 29 C6 EC B3 25                                    )...%

00000000: 90 00                                             ..


----- APDU command/response pair 71 -----
00000000: 80 A4 08 00 08 3F 00 30  00 30 02 00 41           .....?.0.0..A

00000000: 90 00                                             ..


----- APDU command/response pair 72 -----
00000000: 80 48 00 00 00                                    .H...

00000000: \textbf{7F 49 81 88 81} \textbf{81 80} D1  EF 7C A5 06 A1 87 FD 5F  .I.......|.....\_
00000010: 13 5B 25 B7 16 B9 BA A7  21 43 3D DB 51 9D C1 D1  .[\%.....!C=.Q...
00000020: 5A 3C 95 7C B6 F0 37 57  83 CF 2D 0B 53 66 C7 11  Z<.|..7W..-.Sf..
00000030: D5 6B FD 28 FA A0 EA 50  1E 2B FD B5 09 49 E2 E7  .k.(...P.+...I..
00000040: 51 67 1B 00 B0 9D 52 CD  22 D8 69 8C 36 74 54 41  Qg....R.".i.6tTA
00000050: 6E 40 58 4F 79 52 E4 D9  00 43 9C 2C 79 FE A6 48  n@XOyR...C.,y..H
00000060: B7 31 8A B2 05 04 C4 DD  B3 86 E6 4F 38 A6 5D 2A  .1.........O8.]*
00000070: CD A8 3F 95 E4 FF 7B 05  1E ED 4A B5 99 69 36 F0  ..?...\{...J..i6.
00000080: B9 5B 29 C6 EC B3 25 \textbf{82  03} 01 00 01 90 00        .[)...\%.......


----- APDU command/response pair 73 -----
00000000: 80 A4 08 00 08 3F 00 30  00 30 02 01 40           .....?.0.0..@

00000000: 90 00                                             ..


----- APDU command/response pair 74 -----
00000000: 00 D6 00 F5 82 00 80 D1  EF 7C A5 06 A1 87 FD 5F  .........|.....\_
00000010: 13 5B 25 B7 16 B9 BA A7  21 43 3D DB 51 9D C1 D1  .[\%.....!C=.Q...
00000020: 5A 3C 95 7C B6 F0 37 57  83 CF 2D 0B 53 66 C7 11  Z<.|..7W..-.Sf..
00000030: D5 6B FD 28 FA A0 EA 50  1E 2B FD B5 09 49 E2 E7  .k.(...P.+...I..
00000040: 51 67 1B 00 B0 9D 52 CD  22 D8 69 8C 36 74 54 41  Qg....R.".i.6tTA
00000050: 6E 40 58 4F 79 52 E4 D9  00 43 9C 2C 79 FE A6 48  n@XOyR...C.,y..H
00000060: B7 31 8A B2 05 04 C4 DD  B3 86 E6 4F 38 A6 5D 2A  .1.........O8.]*
00000070: CD A8 3F 95 E4 FF 7B 05  1E ED 4A B5 99 69 36 F0  ..?...\{...J..i6.
00000080: B9 5B 29 C6 EC B3 25                              .[)...\%

00000000: 90 00                                             ..

\end{Verbatim}
%\pagebreak



\section{C\_destroyObject}
\begin{Verbatim}[commandchars=\\\{\}, fontsize=\small]
----- APDU command/response pair 35 -----
00000000: 80 A4 08 00 08 3F 00 30  00 30 \textbf{01 03 40}           .....?.0.0..@

00000000: 90 00                                             ..

----- APDU command/response pair 36 -----
00000000: 00 B0 00 00 00                                    .....

00000000: 00 03 \textbf{03 40 01} 23 18 00  00 00 00 04 04 00 00 00  ...@.\#..........
00000010: 00 01 00 00 01 01 00 02  00 00 01 00 00 03 10 00  ................
00000020: 04 64 65 73 33 FF FF FF  FF FF FF FF FF FF FF FF  .des3...........
00000030: FF FF FF FF FF FF FF FF  FF FF FF FF FF FF FF FF  ................
00000040: FF FF FF FF FF FF FF FF  FF FF FF FF FF FF FF FF  ................
00000050: FF FF FF FF FF FF FF FF  FF FF FF FF FF FF FF FF  ................
00000060: FF 00 11 01 00 18 FF FF  FF FF FF FF FF FF FF FF  ................
00000070: FF FF FF FF FF FF FF FF  FF FF FF FF FF FF 01 00  ................
00000080: 00 00 04 15 00 00 00 01  02 10 00 01 01 FF FF FF  ................
00000090: FF FF FF FF FF FF FF FF  FF FF FF FF FF FF FF FF  ................
000000A0: FF FF FF FF FF FF FF FF  FF FF FF FF 01 03 30 00  ..............0.
000000B0: 01 01 01 04 00 00 01 01  01 05 50 00 01 01 01 06  ..........P.....
000000C0: 00 00 01 00 01 07 50 00  01 01 01 08 50 00 01 01  ......P.....P...
000000D0: 01 0A 00 00 01 01 01 0C  10 00 01 00 01 10 10 00  ................
000000E0: 00 FF FF FF FF FF FF FF  FF 01 11 10 00 00 FF FF  ................
000000F0: FF FF FF FF FF FF 01 62  50 00 01 00 01 63 00 00  .......bP....c..
00000100: 61 27                                             a'


----- APDU command/response pair 37 -----
00000000: 00 B0 01 00 00                                    .....

00000000: 01 01 01 64 00 00 01 01  01 65 00 00 01 01 01 66  ...d.....e.....f
00000010: 00 00 04 31 01 00 00 01  70 00 00 01 01 80 10 00  ...1....p.......
00000020: 00 01 00 \textbf{97 03 97 03} 90  00                       .........

----- APDU command/response pair 49 -----
00000000: 80 A4 08 00 08 3F 00 30  00 30 \textbf{03 40 01}           .....?.0.0.@.

00000000: 90 00 

----- APDU command/response pair 50 -----
00000000: 00 D6 00 04 04 \textbf{98 03 98  03}                       .........

00000000: 90 00                                             ..

----- APDU command/response pair 53 -----
00000000: 80 A4 08 00 08 3F 00 30  00 30 \textbf{01 00 C1}           .....?.0.0...

00000000: 90 00                                             ..


----- APDU command/response pair 54 -----
00000000: 00 E4 00 00                                       ....

00000000: 90 00                                             ..


----- APDU command/response pair 55 -----
00000000: 80 A4 08 00 08 3F 00 30  00 30 \textbf{01 03 40}           .....?.0.0..@

00000000: 90 00                                             ..


----- APDU command/response pair 56 -----
00000000: 00 E4 00 00                                       ....

00000000: 90 00                                             ..
\end{Verbatim}
%\pagebreak

\section{C\_encrypt}
\begin{Verbatim}[commandchars=\\\{\}, fontsize=\small]
----- APDU command/response pair 52 -----
00000000: 80 A4 08 00 08 3F 00 30  00 30 01 00 C1           .....?.0.0...

00000000: 90 00                                             ..


----- APDU command/response pair 53 -----
00000000: 00 2A 82 05 13 80 81 10  54 65 73 74 53 74 72 69  .*......TestStri
00000010: 6E 67 31 32 33 34 35 36  00                       ng123456.

00000000: 82 10 B4 F0 97 B6 63 E4  68 7A 8B 00 4F DF 3A C1  ......c.hz..O.:.
00000010: 49 9F 90 00                                       I...
\end{Verbatim}
%\pagebreak

\section{C\_decrypt}
\begin{Verbatim}[commandchars=\\\{\}, fontsize=\small]
----- APDU command/response pair 64 -----
00000000: 80 A4 08 00 08 3F 00 30  00 30 01 00 C1           .....?.0.0...

00000000: 90 00                                             ..


----- APDU command/response pair 65 -----
00000000: 00 2A 80 05 0B 82 81 08  8B 00 4F DF 3A C1 49 9F  .*........O.:.I.
00000010: 00                                                .

00000000: 80 08 6E 67 31 32 33 34  35 36 90 00              ..ng123456..
\end{Verbatim}
%\pagebreak


\section{C\_setAttribute}
\begin{Verbatim}[commandchars=\\\{\}, fontsize=\small]
----- APDU command/response pair 51 -----
00000000: 80 A4 08 00 08 3F 00 30  00 30 01 03 40           .....?.0.0..@

00000000: 90 00                                             ..


----- APDU command/response pair 52 -----
00000000: 90 32 00 03 FF 01 27 00  00 62 82 01 27 00 03 03  .2....'..b..'...
00000010: 40 01 23 18 00 00 00 00  04 04 00 00 00 00 01 00  @.\#.............
00000020: 00 01 01 00 02 00 00 01  00 00 03 10 00 07 63 68  ..............ch
00000030: 61 6E 67 65 64 FF FF FF  FF FF FF FF FF FF FF FF  anged...........
00000040: FF FF FF FF FF FF FF FF  FF FF FF FF FF FF FF FF  ................
00000050: FF FF FF FF FF FF FF FF  FF FF FF FF FF FF FF FF  ................
00000060: FF FF FF FF FF FF FF FF  FF FF FF FF FF FF 00 11  ................
00000070: 01 00 18 FF FF FF FF FF  FF FF FF FF FF FF FF FF  ................
00000080: FF FF FF FF FF FF FF FF  FF FF FF 01 00 00 00 04  ................
00000090: 15 00 00 00 01 02 10 00  01 01 FF FF FF FF FF FF  ................
000000A0: FF FF FF FF FF FF FF FF  FF FF FF FF FF FF FF FF  ................
000000B0: FF FF FF FF FF FF FF FF  FF 01 03 30 00 01 01 01  ...........0....
000000C0: 04 00 00 01 01 01 05 50  00 01 01 01 06 00 00 01  .......P........
000000D0: 00 01 07 50 00 01 01 01  08 50 00 01 01 01 0A 00  ...P.....P......
000000E0: 00 01 01 01 0C 10 00 01  00 01 10 10 00 00 FF FF  ................
000000F0: FF FF FF FF FF FF 01 11  10 00 00 FF FF FF FF FF  ................
00000100: FF FF FF 01                                       ....

00000000: 90 00                                             ..


----- APDU command/response pair 53 -----
00000000: 80 32 00 03 30 62 50 00  01 00 01 63 00 00 01 01  .2..0bP....c....
00000010: 01 64 00 00 01 01 01 65  00 00 01 01 01 66 00 00  .d.....e.....f..
00000020: 04 31 01 00 00 01 70 00  00 01 01 80 10 00 00 01  .1....p.........
00000030: 00 99 03 99 03                                    .....

00000000: 90 00                                             ..

\end{Verbatim}
%\pagebreak

\section{C\_unwrap}
\begin{Verbatim}[commandchars=\\\{\}, fontsize=\small]
----- APDU command/response pair 92 -----
00000000: 80 48 00 80 00                                    .H...

00000000: 80 01 05 81 81 80 F7 B5  15 72 07 22 94 6F C4 08  .........r.".o..
00000010: 64 CB BD AF EA 55 7D BD  8F 55 36 B0 01 C2 8B 2E  d....U\}..U6.....
00000020: 32 B6 5D 45 F1 74 5D 38  12 0B AD 9D 2C 03 9C 22  2.]E.t]8....,.."
00000030: 46 68 EB 2E A2 8C 20 95  A8 2E 6C A8 E0 6D 47 F2  Fh.... ...l..mG.
00000040: D3 1E D7 01 F8 15 5C AD  DC 05 70 C0 93 B2 6D 74  ......\...p...mt
00000050: B0 9B 95 E6 4D 8C D2 FC  73 3E CD 0F 30 68 79 A5  ....M...s>..0hy.
00000060: B9 35 F2 41 3F 52 AD AD  32 A0 99 1A 18 3D CC 57  .5.A?R..2....=.W
00000070: 7E 39 DA 47 53 1E 67 15  AB 01 70 7F F2 47 96 71  ~9.GS.g...p..G.q
00000080: 44 23 CE 7B 60 67 82 81  80 3C 52 D2 06 89 28 92  D#.\{`g...<R...(.
00000090: 2C AB E6 3C 4E E6 DF 0E  D2 29 F1 01 BE 36 C4 F8  ,..<N....)...6..
000000A0: 54 40 56 F3 4A FA 8D 2E  9B 60 F5 07 BC ED B4 44  T@V.J....`.....D
000000B0: 56 68 5D 82 4C C4 EA D7  96 20 F8 C5 46 A6 E0 16  Vh].L.... ..F...
000000C0: B8 AB A5 D8 43 29 58 53  77 17 09 97 AA 70 68 33  ....C)XSw....ph3
000000D0: 9E F1 41 0A 5F 39 D9 75  24 7F 3A 53 63 61 47 87  ..A.\_9.u\$.:ScaG.
000000E0: 87 7F 88 96 BC BB 83 A1  CB D1 42 E0 EB 99 CF 34  ..........B....4
000000F0: 0E CA 56 4F 2C 57 50 6E  7B 1A FC 1F 90 7A E0 C2  ..VO,WPn\{....z..
00000100: 61 09                                             a.


----- APDU command/response pair 93 -----
00000000: 00 C0 00 00 09                                    .....

00000000: A8 5D D3 30 E3 5C A9 00  39 90 00                 .].0....9..


----- APDU command/response pair 94 -----
00000000: 80 86 00 00 80 D0 7E EE  17 C7 31 DD 53 FB 1F D4  ......~...1.S...
00000010: 36 65 EB 7F 2C B0 A2 34  44 80 D7 F4 31 96 12 DF  6e..,..4D...1...
00000020: C8 AD 3C 41 EE 8F 13 C2  8A 3B 8D 6B 73 18 A6 1B  ..<A.....;.ks...
00000030: 46 3E 10 93 5C 2F 35 1C  A3 FC 48 09 DB E4 BB EA  F>..\/5...H.....
00000040: 3F 1A 11 7D 85 57 2F 85  75 1D 8B F4 E6 39 2B FA  ?..\}.W/.u....9+.
00000050: 19 3D 7A BB E3 75 B2 A2  A9 E4 EE 79 4F A6 3F EE  .=z..u.....yO.?.
00000060: FD BF 4A F8 43 8F DA A9  D1 8D 58 63 12 5D C8 E8  ..J.C.....Xc.]..
00000070: 2C 77 8F 5F 96 C0 51 CA  19 B1 80 D5 80 4E 50 8B  ,w.\_..Q......NP.
00000080: 88 6B 64 43 0D 00                                 .kdC..

00000000: FD 74 3B 38 48 7E 0E D9  4D B0 BF E7 66 3D E4 63  .t;8H~..M...f=.c
00000010: 15 24 EC 7B F3 93 C7 90  85 43 E8 DF D9 E0 60 88  .\$.\{.....C....`.
00000020: 90 00                                             ..


----- APDU command/response pair 95 -----
00000000: 8C A4 08 00 1D 87 11 01  65 FD 9A B5 09 70 96 93  ........e....p..
00000010: FB 5D 39 FF B3 24 6B 8E  8E 08 04 D7 B0 58 E0 96  .]9..\$k......X..
00000020: E6 01                                             ..

00000000: 99 02 90 00 8E 08 67 C9  1F 50 18 5F 6D 6A 90 00  ......g..P.\_mj..


----- APDU command/response pair 96 -----
00000000: 0C 2A 80 0A 99 87 81 89  01 1D 7A 97 D8 25 8F 60  .*........z..\%.`
00000010: 52 07 AE DC A9 AC 33 7C  6E 12 A9 79 71 B8 36 1B  R.....3|n..yq.6.
00000020: 29 C3 54 C1 A8 29 A4 4F  75 72 4E C6 C5 71 22 88  ).T..).OurN..q".
00000030: 50 0C 29 9F 75 C7 99 39  E9 B6 5B AF A1 65 51 DE  P.).u..9..[..eQ.
00000040: 56 84 6D 30 B6 2F F3 19  6B 83 82 C4 6B AB 59 E3  V.m0./..k...k.Y.
00000050: 2B FD B1 4B FC 3D BE CD  16 C8 C0 69 80 5C 0E 72  +..K.=.....i...r
00000060: C0 0F 24 0A 3E 8A 88 4A  CA 68 02 5C FA B5 36 33  ..\$.>..J.h.\..63
00000070: CB 5A F7 BE 86 21 2F 68  DB 5F 46 1D 67 FA C2 8B  .Z...!/h.\_F.g...
00000080: A9 58 37 5C F0 34 7E FE  FC 1A 78 46 C7 51 0B 13  .X7\.4~...xF.Q..
00000090: B2 97 01 00 8E 08 F9 2D  53 7C AD 46 EB 79 00     .......-S|.F.y.

00000000: 87 11 01 FC E7 96 A5 B1  96 E9 E3 1D 2D 3A 49 46  ............-:IF
00000010: 8C 97 A7 99 02 90 00 8E  08 41 16 94 D4 58 27 D0  .........A...X'.
00000020: 3F 90 00                                          ?..


----- APDU command/response pair 97 -----
00000000: 80 86 FF FF                                       ....

00000000: 90 00                                             ..


----- APDU command/response pair 115 -----
00000000: 80 A4 08 00 08 3F 00 30  00 30 03 40 01           .....?.0.0.@.

00000000: 90 00                                             ..


----- APDU command/response pair 116 -----
00000000: 00 D6 00 04 04 B0 03 B0  03                       .........

00000000: 90 00                                             ..


----- APDU command/response pair 117 -----
00000000: 80 A4 08 00 06 3F 00 30  00 30 01                 .....?.0.0.

00000000: 90 00                                             ..


----- APDU command/response pair 118 -----
00000000: 00 E0 01 00 18 62 81 15  8A 01 04 83 02 03 41 80  .....b........A.
00000010: 02 01 27 86 08 00 00 00  00 00 00 00 00           ..'..........

00000000: 90 00                                             ..


----- APDU command/response pair 119 -----
00000000: 80 A4 08 00 08 3F 00 30  00 30 01 03 41           .....?.0.0..A

00000000: 90 00                                             ..


----- APDU command/response pair 120 -----
00000000: 00 D6 00 00 FA 01 03 03  41 01 23 18 00 00 00 00  ........A.\#.....
00000010: 04 04 00 00 00 00 01 00  00 01 01 00 02 00 00 01  ................
00000020: 00 00 03 10 00 04 74 65  73 74 FF FF FF FF FF FF  ......test......
00000030: FF FF FF FF FF FF FF FF  FF FF FF FF FF FF FF FF  ................
00000040: FF FF FF FF FF FF FF FF  FF FF FF FF FF FF FF FF  ................
00000050: FF FF FF FF FF FF FF FF  FF FF FF FF FF FF FF FF  ................
00000060: FF FF FF FF FF FF 00 11  01 00 08 FF FF FF FF FF  ................
00000070: FF FF FF FF FF FF FF FF  FF FF FF FF FF FF FF FF  ................
00000080: FF FF FF 01 00 00 00 04  13 00 00 00 01 02 10 00  ................
00000090: 01 10 FF FF FF FF FF FF  FF FF FF FF FF FF FF FF  ................
000000A0: FF FF FF FF FF FF FF FF  FF FF FF FF FF FF FF FF  ................
000000B0: FF 01 03 30 00 01 01 01  04 00 00 01 01 01 05 50  ...0...........P
000000C0: 00 01 01 01 06 00 00 01  00 01 07 50 00 01 01 01  ...........P....
000000D0: 08 50 00 01 01 01 0A 00  00 01 01 01 0C 10 00 01  .P..............
000000E0: 00 01 10 10 00 00 FF FF  FF FF FF FF FF FF 01 11  ................
000000F0: 10 00 00 FF FF FF FF FF  FF FF FF 01 62 50 00     ............bP.

00000000: 90 00                                             ..


----- APDU command/response pair 121 -----
00000000: 00 D6 00 FA 2D 01 00 01  63 00 00 01 00 01 64 00  ....-...c.....d.
00000010: 00 01 00 01 65 00 00 01  00 01 66 00 00 04 FF FF  ....e.....f.....
00000020: FF FF 01 70 00 00 01 01  80 10 00 00 01 00 B0 03  ...p............
00000030: B0 03                                             ..

00000000: 90 00                                             ..


----- APDU command/response pair 122 -----
00000000: 80 48 00 80 00                                    .H...

00000000: 80 01 05 81 81 80 F7 B5  15 72 07 22 94 6F C4 08  .........r.".o..
00000010: 64 CB BD AF EA 55 7D BD  8F 55 36 B0 01 C2 8B 2E  d....U\}..U6.....
00000020: 32 B6 5D 45 F1 74 5D 38  12 0B AD 9D 2C 03 9C 22  2.]E.t]8....,.."
00000030: 46 68 EB 2E A2 8C 20 95  A8 2E 6C A8 E0 6D 47 F2  Fh.... ...l..mG.
00000040: D3 1E D7 01 F8 15 5C AD  DC 05 70 C0 93 B2 6D 74  ......\...p...mt
00000050: B0 9B 95 E6 4D 8C D2 FC  73 3E CD 0F 30 68 79 A5  ....M...s>..0hy.
00000060: B9 35 F2 41 3F 52 AD AD  32 A0 99 1A 18 3D CC 57  .5.A?R..2....=.W
00000070: 7E 39 DA 47 53 1E 67 15  AB 01 70 7F F2 47 96 71  ~9.GS.g...p..G.q
00000080: 44 23 CE 7B 60 67 82 81  80 3C 52 D2 06 89 28 92  D\#.\{`g...<R...(.
00000090: 2C AB E6 3C 4E E6 DF 0E  D2 29 F1 01 BE 36 C4 F8  ,..<N....)...6..
000000A0: 54 40 56 F3 4A FA 8D 2E  9B 60 F5 07 BC ED B4 44  T@V.J....`.....D
000000B0: 56 68 5D 82 4C C4 EA D7  96 20 F8 C5 46 A6 E0 16  Vh].L.... ..F...
000000C0: B8 AB A5 D8 43 29 58 53  77 17 09 97 AA 70 68 33  ....C)XSw....ph3
000000D0: 9E F1 41 0A 5F 39 D9 75  24 7F 3A 53 63 61 47 87  ..A.\_9.u\$.:ScaG.
000000E0: 87 7F 88 96 BC BB 83 A1  CB D1 42 E0 EB 99 CF 34  ..........B....4
000000F0: 0E CA 56 4F 2C 57 50 6E  7B 1A FC 1F 90 7A E0 C2  ..VO,WPn\{....z..
00000100: 61 09                                             a.


----- APDU command/response pair 123 -----
00000000: 00 C0 00 00 09                                    .....

00000000: A8 5D D3 30 E3 5C A9 00  39 90 00                 .].0....9..


----- APDU command/response pair 124 -----
00000000: 80 86 00 00 80 95 3B CF  46 B8 4E 67 E4 6B 97 4B  ......;.F.Ng.k.K
00000010: 70 AD B3 44 22 6A 1B 42  18 4B A9 44 FF 28 FA C0  p..D"j.B.K.D.(..
00000020: 0A EF 44 CD DA C1 28 2B  CF FD 5D 20 48 50 33 59  ..D...(+..] HP3Y
00000030: 7D B7 CB 73 4A EF 28 0A  C7 E4 02 2A 91 A9 F6 55  \}..sJ.(....*...U
00000040: 97 D3 A8 DE 21 90 0E 23  0B 9C ED 4B 52 39 46 ED  ....!..\#...KR9F.
00000050: 13 1F 7F 9D CB EF 7A DD  7C D7 39 EC 1F BD 2A 3A  ......z.|.9...*:
00000060: 45 48 8F 6C 7E 82 71 E5  14 8F C1 9D F8 E8 53 2B  EH.l~.q.......S+
00000070: D3 AF 3D 7C 11 59 E3 81  F4 0B 08 17 A9 0F 37 69  ..=|.Y........7i
00000080: 90 C1 11 E2 1B 00                                 ......

00000000: B3 0F 6C 66 E6 56 8F 44  55 B2 A6 02 0E 0B 80 01  ..lf.V.DU.......
00000010: FF 89 7A 65 FC 68 25 82  22 C9 97 74 D1 6B 00 AB  ..ze.h\%."..t.k..
00000020: 90 00                                             ..


----- APDU command/response pair 125 -----
00000000: 8C A4 08 00 15 87 09 01  82 46 BD FD 60 2D E4 C6  .........F..`-..
00000010: 8E 08 25 35 C0 28 0E E1  20 93                    ..\%5.(.. .

00000000: 99 02 90 00 8E 08 8C D6  A9 A8 99 7F 14 12 90 00  ................


----- APDU command/response pair 126 -----
00000000: 0C E0 08 00 3D 87 31 01  4D 4F 3D AB 31 72 FC F7  ....=.1.MO=.1r..
00000010: B4 84 D1 41 19 1C 22 DF  3F 60 BE 6B 0A 1E 49 5F  ...A..".?`.k..I_
00000020: AD 3D 6D 61 5E DA E3 F7  A8 0A 82 EA 65 16 8A 01  .=ma^.......e...
00000030: C5 4F BF 3F 44 73 9C 61  8E 08 A8 A9 A5 4D 55 BB  .O.?Ds.a.....MU.
00000040: E7 B3                                             ..

00000000: 99 02 90 00 8E 08 23 E5  DF 34 11 21 87 1C 90 00  ......\#..4.!....
\end{Verbatim}
%\pagebreak

\section{C\_wrap}
\begin{Verbatim}[commandchars=\\\{\}, fontsize=\small]
Enter command (spaced integers)
00 42 130 10 08 30 31 32 34 35 36 37 38 00 (integers)
00 2a 82  0a 08 1e 1f 20 22 23 24 25 26 00 (hexadecimals)
command changed!

RESPONSE
00000000: 6A 80                                             j.
\end{Verbatim}

\end{appendices} 
\end{document}


